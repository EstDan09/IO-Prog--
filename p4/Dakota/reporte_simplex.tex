\documentclass[11pt]{article}
\usepackage[spanish]{babel}
\usepackage[utf8]{inputenc}
\usepackage[T1]{fontenc}
\usepackage[a4paper,margin=2.2cm]{geometry}
\usepackage{amsmath,amssymb,booktabs,array,colortbl}
\usepackage[table,xcdraw]{xcolor}
\usepackage{hyperref,graphicx}
\begin{document}
\begin{titlepage}
\centering
{\Huge Proyecto 4 - Otro S\'implex M\'as}\\[1em]
{\Large Dakota}\\[2em]
{\large Curso: Investigaci\'on de Operaciones\\Semestre: 2025-I}\\[4em]
\vfill
\textbf{Esteban Secaida - Fabian Bustos}\\[2em]
Fecha: \today
\end{titlepage}
\newpage
\section*{Planteamiento del Problema}
Maximizar \[ Z = 60.000Escritorios +30.000Mesas +20.000Sillas \]
Sujeto a:\[ 
8.000Escritorios 
+6.000Mesas 
+1.000Sillas 
\le 48.000\\
4.000Escritorios 
+2.000Mesas 
+1.500Sillas 
\le 20.000\\
2.000Escritorios 
+1.500Mesas 
+0.500Sillas 
\le 8.000\\
0.000Escritorios 
+1.000Mesas 
+0.000Sillas 
\le 5.000\\
x_i \ge 0 \text{ para todo } i.\]
\section*{Descripci\'on del M\'etodo S\'implex}
El algoritmo S\'implex, propuesto por George Dantzig en 1947, es un procedimiento iterativo que explora los v\'ertices del poliedro factible para encontrar la soluci\'on \textit{\'optima} de un problema lineal. En cada iteraci\'on se determina una variable que entra a la base y otra que sale, hasta que no existen mejoras posibles en la funci\'on objetivo.\\[1em]
\section*{Tablas del M\'etodo S\'implex}
\begin{table}[h]
\centering
\caption{Tabla inicial.}
\setlength{\tabcolsep}{6pt}
\renewcommand{\arraystretch}{1.15}
\begin{tabular}{lrrrrrrrr}
\toprule
 & $x_{1}$ & $x_{2}$ & $x_{3}$ & $s_{1}$ & $s_{2}$ & $s_{3}$ & $s_{4}$ & $b$ \\
\midrule
$Z$ & -60.000000 & -30.000000 & -20.000000 & 0.000000 & 0.000000 & 0.000000 & 0.000000 & 0.000000 \\
$R_{1}$ & 8.000000 & 6.000000 & 1.000000 & 1.000000 & 0.000000 & 0.000000 & 0.000000 & 48.000000 \\
$R_{2}$ & 4.000000 & 2.000000 & 1.500000 & 0.000000 & 1.000000 & 0.000000 & 0.000000 & 20.000000 \\
$R_{3}$ & 2.000000 & 1.500000 & 0.500000 & 0.000000 & 0.000000 & 1.000000 & 0.000000 & 8.000000 \\
$R_{4}$ & 0.000000 & 1.000000 & 0.000000 & 0.000000 & 0.000000 & 0.000000 & 1.000000 & 5.000000 \\
\bottomrule
\end{tabular}
\end{table}

\begin{table}[h]
\centering
\caption{Iteraci\'on 1: entra la columna $x_{1}$ y sale la fila $R_{3}$.}
\setlength{\tabcolsep}{6pt}
\renewcommand{\arraystretch}{1.15}
\begin{tabular}{lrrrrrrrr}
\toprule
 & $x_{1}$ & $x_{2}$ & $x_{3}$ & $s_{1}$ & $s_{2}$ & $s_{3}$ & $s_{4}$ & $b$ \\
\midrule
$Z$ & \cellcolor{blue!12}-60.000000 & -30.000000 & -20.000000 & 0.000000 & 0.000000 & 0.000000 & 0.000000 & 0.000000 \\
$R_{1}$ & \cellcolor{blue!12}8.000000 & 6.000000 & 1.000000 & 1.000000 & 0.000000 & 0.000000 & 0.000000 & 48.000000 \\
$R_{2}$ & \cellcolor{blue!12}4.000000 & 2.000000 & 1.500000 & 0.000000 & 1.000000 & 0.000000 & 0.000000 & 20.000000 \\
$R_{3}$ & \cellcolor{orange!35}2.000000 & \cellcolor{green!12}1.500000 & \cellcolor{green!12}0.500000 & \cellcolor{green!12}0.000000 & \cellcolor{green!12}0.000000 & \cellcolor{green!12}1.000000 & \cellcolor{green!12}0.000000 & \cellcolor{green!12}8.000000 \\
$R_{4}$ & \cellcolor{blue!12}0.000000 & 1.000000 & 0.000000 & 0.000000 & 0.000000 & 0.000000 & 1.000000 & 5.000000 \\
\bottomrule
\end{tabular}
\\[4pt]\textbf{Fracciones } $b_i/a_{i,j}$ para la columna $x_{1}$:\\
$R_{1} = 6.000000$,\ $R_{2} = 5.000000$,\ $R_{3} = 4.000000$ \;\;\textbf{(m\'inima)},\ .
\end{table}

\begin{table}[h]
\centering
\caption{Iteraci\'on 2: entra la columna $x_{3}$ y sale la fila $R_{2}$.}
\setlength{\tabcolsep}{6pt}
\renewcommand{\arraystretch}{1.15}
\begin{tabular}{lrrrrrrrr}
\toprule
 & $x_{1}$ & $x_{2}$ & $x_{3}$ & $s_{1}$ & $s_{2}$ & $s_{3}$ & $s_{4}$ & $b$ \\
\midrule
$Z$ & 0.000000 & 15.000000 & \cellcolor{blue!12}-5.000000 & 0.000000 & 0.000000 & 30.000000 & 0.000000 & 240.000000 \\
$R_{1}$ & 0.000000 & 0.000000 & \cellcolor{blue!12}-1.000000 & 1.000000 & 0.000000 & -4.000000 & 0.000000 & 16.000000 \\
$R_{2}$ & \cellcolor{green!12}0.000000 & \cellcolor{green!12}-1.000000 & \cellcolor{orange!35}0.500000 & \cellcolor{green!12}0.000000 & \cellcolor{green!12}1.000000 & \cellcolor{green!12}-2.000000 & \cellcolor{green!12}0.000000 & \cellcolor{green!12}4.000000 \\
$R_{3}$ & 1.000000 & 0.750000 & \cellcolor{blue!12}0.250000 & 0.000000 & 0.000000 & 0.500000 & 0.000000 & 4.000000 \\
$R_{4}$ & 0.000000 & 1.000000 & \cellcolor{blue!12}0.000000 & 0.000000 & 0.000000 & 0.000000 & 1.000000 & 5.000000 \\
\bottomrule
\end{tabular}
\\[4pt]\textbf{Fracciones } $b_i/a_{i,j}$ para la columna $x_{3}$:\\
$R_{2} = 8.000000$ \;\;\textbf{(m\'inima)},\ $R_{3} = 16.000000$,\ .
\end{table}

\begin{table}[h]
\centering
\caption{Tabla final.}
\setlength{\tabcolsep}{6pt}
\renewcommand{\arraystretch}{1.15}
\begin{tabular}{lrrrrrrrr}
\toprule
 & $x_{1}$ & $x_{2}$ & $x_{3}$ & $s_{1}$ & $s_{2}$ & $s_{3}$ & $s_{4}$ & $b$ \\
\midrule
$Z$ & 0.000000 & 5.000000 & 0.000000 & 0.000000 & 10.000000 & 10.000000 & 0.000000 & 280.000000 \\
$R_{1}$ & 0.000000 & -2.000000 & 0.000000 & 1.000000 & 2.000000 & -8.000000 & 0.000000 & 24.000000 \\
$R_{2}$ & 0.000000 & -2.000000 & 1.000000 & 0.000000 & 2.000000 & -4.000000 & 0.000000 & 8.000000 \\
$R_{3}$ & 1.000000 & 1.250000 & 0.000000 & 0.000000 & -0.500000 & 1.500000 & 0.000000 & 2.000000 \\
$R_{4}$ & 0.000000 & 1.000000 & 0.000000 & 0.000000 & 0.000000 & 0.000000 & 1.000000 & 5.000000 \\
\bottomrule
\end{tabular}
\end{table}

\section*{Resultados y Casos Especiales}
Estado del problema: \textbf{Óptimo}.\\
Valor \textit{\'optimo}: $Z^* = 280.000000$.\\[4pt]
Soluci\'on \textit{\'optima}:\\[4pt]
\[ Escritorios = 2.000000,\;Mesas = 0.000000,\;Sillas = 8.000000. \]
\vfill\smallskip\noindent Documento generado autom\'aticamente por \texttt{simplex\_report.c}.\\
\end{document}
