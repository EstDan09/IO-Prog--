\documentclass[11pt]{article}
\usepackage[spanish]{babel}
\usepackage[utf8]{inputenc}
\usepackage[T1]{fontenc}
\usepackage[a4paper,margin=2.2cm]{geometry}
\usepackage{amsmath,amssymb,booktabs,array,colortbl}
\usepackage[table,xcdraw]{xcolor}
\usepackage{hyperref,graphicx}
\begin{document}
\begin{titlepage}
\centering
{\Huge Proyecto 4 - Otro S\'implex M\'as}\\[1em]
{\Large ALOHA}\\[2em]
{\large Curso: Investigaci\'on de Operaciones\\Semestre: 2025-I}\\[4em]
\vfill
\textbf{Esteban Secaida - Fabian Bustos}\\[2em]
Fecha: \today
\end{titlepage}
\newpage
\section*{Planteamiento del Problema}
Maximizar \[ Z = 1.000x1 +1.000x2 \]
Sujeto a:\[ 
1.000x1 +1.000x2 \le 1.000\\
1.000x1 +1.000x2 \le 1.000\\
x_i \ge 0 \text{ para todo } i.\]
\section*{Descripci\'on del M\'etodo S\'implex}
El algoritmo S\'implex, propuesto por George Dantzig en 1947, es un procedimiento iterativo que explora los v\'ertices del poliedro factible para encontrar la soluci\'on \textit{\'optima} de un problema lineal. En cada iteraci\'on se determina una variable que entra a la base y otra que sale, hasta que no existen mejoras posibles en la funci\'on objetivo.\\[1em]
\section*{Tablas del M\'etodo S\'implex}
\begin{table}[h]
\centering
\caption{Tabla inicial.}
\setlength{\tabcolsep}{6pt}
\renewcommand{\arraystretch}{1.15}
\begin{tabular}{lrrrrr}
\toprule
Base & $x_{1}$ & $x_{2}$ & $y_{1}$ & $y_{2}$ & $b$ \\
\midrule
$Z$ & -1.000000 & -1.000000 & 0.000000 & 0.000000 & 0.000000 \\
$R_{1}$ & 1.000000 & 1.000000 & 1.000000 & 0.000000 & 1.000000 \\
$R_{2}$ & 1.000000 & 1.000000 & 0.000000 & 1.000000 & 1.000000 \\
\bottomrule
\end{tabular}
\end{table}

\begin{table}[h]
\centering
\caption{Iteraci\'on 1: entra la columna $x_{1}$ y sale la fila $R_{1}$.}
\setlength{\tabcolsep}{6pt}
\renewcommand{\arraystretch}{1.15}
\begin{tabular}{lrrrrr}
\toprule
Base & $x_{1}$ & $x_{2}$ & $y_{1}$ & $y_{2}$ & $b$ \\
\midrule
$Z$ & \cellcolor{blue!12}-1.000000 & -1.000000 & 0.000000 & 0.000000 & 0.000000 \\
$R_{1}$ & \cellcolor{orange!35}1.000000 & \cellcolor{green!12}1.000000 & \cellcolor{green!12}1.000000 & \cellcolor{green!12}0.000000 & \cellcolor{green!12}1.000000 \\
$R_{2}$ & \cellcolor{blue!12}1.000000 & 1.000000 & 0.000000 & 1.000000 & 1.000000 \\
\bottomrule
\end{tabular}
\\[4pt]\textbf{Fracciones } $b_i/a_{i,j}$ para la columna $x_{1}$:\\
$R_{1} = 1.000000$ \;\;\textbf{(m\'inima)},\ $R_{2} = 1.000000$.
\end{table}

\begin{table}[h]
\centering
\caption{Tabla final.}
\setlength{\tabcolsep}{6pt}
\renewcommand{\arraystretch}{1.15}
\begin{tabular}{lrrrrr}
\toprule
Base & $x_{1}$ & $x_{2}$ & $y_{1}$ & $y_{2}$ & $b$ \\
\midrule
$Z$ & 0.000000 & 0.000000 & 1.000000 & 0.000000 & 1.000000 \\
$R_{1}$ & 1.000000 & 1.000000 & 1.000000 & 0.000000 & 1.000000 \\
$R_{2}$ & 0.000000 & 0.000000 & -1.000000 & 1.000000 & 0.000000 \\
\bottomrule
\end{tabular}
\end{table}

\section*{Resultados y Casos Especiales}
Estado del problema: \textbf{Óptimo (múltiples soluciones)}.\\
Valor \textit{\'optimo}: $Z^* = 1.000000$.\\[4pt]
Soluci\'on \textit{\'optima}:\\[4pt]
\[ x1 = 1.000000,\;x2 = 0.000000. \]
El problema presenta \textbf{m\'ultiples soluciones \textit{\'optimas}}. Se puede obtener una familia de soluciones a lo largo de la recta de \textit{\'optimos}.\\[6pt]
\subsection*{M\'ultiples soluciones}
Se detect\'o una variable no b\'asica con costo reducido cero: $x_{2}$. Esto implica la existencia de un conjunto infinito de \'{o}ptimos.\\[4pt]
Una parametrizaci\'on de la arista \'optima es:
$$$x_{2}$ = t,\quad x_{B_i} = b_i - a_{i,2}\,t,\quad 0 \le t \le \theta,$$ tomando $\theta = 1.000000$ de las fracciones v\'alidas.\\[6pt]
Punto 1: $t=0.250000$ \;$\Rightarrow$\; $x_{1}=0.750000$,\ $x_{2}=0.250000$.\\
Punto 2: $t=0.500000$ \;$\Rightarrow$\; $x_{1}=0.500000$,\ $x_{2}=0.500000$.\\
Punto 3: $t=0.750000$ \;$\Rightarrow$\; $x_{1}=0.250000$,\ $x_{2}=0.750000$.\\
\paragraph{Tabla final alterna.}
\begin{table}[h]
\centering
\caption{Tabla final.}
\setlength{\tabcolsep}{6pt}
\renewcommand{\arraystretch}{1.15}
\begin{tabular}{lrrrrr}
\toprule
Base & $x_{1}$ & $x_{2}$ & $y_{1}$ & $y_{2}$ & $b$ \\
\midrule
$Z$ & 0.000000 & 0.000000 & 1.000000 & 0.000000 & 1.000000 \\
$R_{1}$ & 1.000000 & 1.000000 & 1.000000 & 0.000000 & 1.000000 \\
$R_{2}$ & 0.000000 & 0.000000 & -1.000000 & 1.000000 & 0.000000 \\
\bottomrule
\end{tabular}
\end{table}

Soluci\'on alterna b\'asica: $x_{1}=0.000000$,\ $x_{2}=1.000000$;\quad Z = 1.000000.\\

\vfill\smallskip\noindent Documento generado por Otro Simplex mas, de Esteban Secaida y Fabi\'an Bustos \end{document}
