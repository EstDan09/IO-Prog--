\documentclass[11pt]{article}
\usepackage[spanish]{babel}
\usepackage[utf8]{inputenc}
\usepackage[T1]{fontenc}
\usepackage[a4paper,margin=2.2cm]{geometry}
\usepackage{amsmath,amssymb,booktabs,array,colortbl}
\usepackage[table,xcdraw]{xcolor}
\usepackage{hyperref,graphicx}
\begin{document}
\begin{titlepage}
\centering
{\Huge Proyecto 4 - Otro S\'implex M\'as}\\[1em]
{\Large Ejemplo_BigM_Min}\\[2em]
{\large Curso: Investigaci\'on de Operaciones\\Semestre: 2025-I}\\[4em]
\vfill
\textbf{Esteban Secaida - Fabian Bustos}\\[2em]
Fecha: \today
\end{titlepage}
\newpage
\section*{Planteamiento del Problema}
Minimizar \[ Z = 1.000x_1 +1.000x_2 \]
Sujeto a:\[ 
1.000x_1 +1.000x_2 \ge 4.000\\
1.000x_1 +0.000x_2 \ge 1.000\\
0.000x_1 +1.000x_2 \le 5.000\\
x_i \ge 0 \text{ para todo } i.\]
\section*{Descripci\'on del M\'etodo S\'implex}
El algoritmo S\'implex, propuesto por George Dantzig en 1947, es un procedimiento iterativo que explora los v\'ertices del poliedro factible para encontrar la soluci\'on \textit{\'optima} de un problema lineal. En cada iteraci\'on se determina una variable que entra a la base y otra que sale, hasta que no existen mejoras posibles en la funci\'on objetivo.\\[1em]
\section*{Tablas del M\'etodo S\'implex}
\begin{table}[h]
\centering
\caption{Tabla inicial.}
\setlength{\tabcolsep}{6pt}
\renewcommand{\arraystretch}{1.15}
\begin{tabular}{lrrrrrr}
\toprule
 & $x_{1}$ & $x_{2}$ & $s_{1}$ & $s_{2}$ & $s_{3}$ & $b$ \\
\midrule
$Z$ & -1999999.000000 & -999999.000000 & 0.000000 & 1000000.000000 & 1000000.000000 & 0.000000 & 0.000000 & -5000000.000000 \\
$R_{1}$ & 1.000000 & 1.000000 & 0.000000 & -1.000000 & 0.000000 & 1.000000 & 0.000000 & 4.000000 \\
$R_{2}$ & 1.000000 & 0.000000 & 0.000000 & 0.000000 & -1.000000 & 0.000000 & 1.000000 & 1.000000 \\
$R_{3}$ & 0.000000 & 1.000000 & 1.000000 & 0.000000 & 0.000000 & 0.000000 & 0.000000 & 5.000000 \\
\bottomrule
\end{tabular}
\end{table}

\begin{table}[h]
\centering
\caption{Iteraci\'on 1: entra la columna $x_{1}$ y sale la fila $R_{2}$.}
\setlength{\tabcolsep}{6pt}
\renewcommand{\arraystretch}{1.15}
\begin{tabular}{lrrrrrr}
\toprule
 & $x_{1}$ & $x_{2}$ & $s_{1}$ & $s_{2}$ & $s_{3}$ & $b$ \\
\midrule
$Z$ & \cellcolor{blue!12}-1999999.000000 & -999999.000000 & 0.000000 & 1000000.000000 & 1000000.000000 & 0.000000 & 0.000000 & -5000000.000000 \\
$R_{1}$ & \cellcolor{blue!12}1.000000 & 1.000000 & 0.000000 & -1.000000 & 0.000000 & 1.000000 & 0.000000 & 4.000000 \\
$R_{2}$ & \cellcolor{orange!35}1.000000 & \cellcolor{green!12}0.000000 & \cellcolor{green!12}0.000000 & \cellcolor{green!12}0.000000 & \cellcolor{green!12}-1.000000 & \cellcolor{green!12}0.000000 & \cellcolor{green!12}1.000000 & \cellcolor{green!12}1.000000 \\
$R_{3}$ & \cellcolor{blue!12}0.000000 & 1.000000 & 1.000000 & 0.000000 & 0.000000 & 0.000000 & 0.000000 & 5.000000 \\
\bottomrule
\end{tabular}
\\[4pt]\textbf{Fracciones } $b_i/a_{i,j}$ para la columna $x_{1}$:\\
$R_{1} = 4.000000$,\ $R_{2} = 1.000000$ \;\;\textbf{(m\'inima)},\ .
\end{table}

\begin{table}[h]
\centering
\caption{Iteraci\'on 2: entra la columna $x_{2}$ y sale la fila $R_{1}$.}
\setlength{\tabcolsep}{6pt}
\renewcommand{\arraystretch}{1.15}
\begin{tabular}{lrrrrrr}
\toprule
 & $x_{1}$ & $x_{2}$ & $s_{1}$ & $s_{2}$ & $s_{3}$ & $b$ \\
\midrule
$Z$ & 0.000000 & \cellcolor{blue!12}-999999.000000 & 0.000000 & 1000000.000000 & -999999.000000 & 0.000000 & 1999999.000000 & -3000001.000000 \\
$R_{1}$ & \cellcolor{green!12}0.000000 & \cellcolor{orange!35}1.000000 & \cellcolor{green!12}0.000000 & \cellcolor{green!12}-1.000000 & \cellcolor{green!12}1.000000 & \cellcolor{green!12}1.000000 & \cellcolor{green!12}-1.000000 & \cellcolor{green!12}3.000000 \\
$R_{2}$ & 1.000000 & \cellcolor{blue!12}0.000000 & 0.000000 & 0.000000 & -1.000000 & 0.000000 & 1.000000 & 1.000000 \\
$R_{3}$ & 0.000000 & \cellcolor{blue!12}1.000000 & 1.000000 & 0.000000 & 0.000000 & 0.000000 & 0.000000 & 5.000000 \\
\bottomrule
\end{tabular}
\\[4pt]\textbf{Fracciones } $b_i/a_{i,j}$ para la columna $x_{2}$:\\
$R_{1} = 3.000000$ \;\;\textbf{(m\'inima)},\ $R_{3} = 5.000000$.
\end{table}

\begin{table}[h]
\centering
\caption{Tabla final.}
\setlength{\tabcolsep}{6pt}
\renewcommand{\arraystretch}{1.15}
\begin{tabular}{lrrrrrr}
\toprule
 & $x_{1}$ & $x_{2}$ & $s_{1}$ & $s_{2}$ & $s_{3}$ & $b$ \\
\midrule
$Z$ & 0.000000 & 0.000000 & 0.000000 & 1.000000 & 0.000000 & 999999.000000 & 1000000.000000 & -4.000000 \\
$R_{1}$ & 0.000000 & 1.000000 & 0.000000 & -1.000000 & 1.000000 & 1.000000 & -1.000000 & 3.000000 \\
$R_{2}$ & 1.000000 & 0.000000 & 0.000000 & 0.000000 & -1.000000 & 0.000000 & 1.000000 & 1.000000 \\
$R_{3}$ & 0.000000 & 0.000000 & 1.000000 & 1.000000 & -1.000000 & -1.000000 & 1.000000 & 2.000000 \\
\bottomrule
\end{tabular}
\end{table}

\section*{Resultados y Casos Especiales}
Estado del problema: \textbf{Óptimo}.\\
Valor \textit{\'optimo}: $Z^* = 4.000000$.\\[4pt]
Soluci\'on \textit{\'optima}:\\[4pt]
\[ x_1 = 1.000000,\;x_2 = 3.000000. \]
\vfill\smallskip\noindent Documento generado por Otro Simplex mas, de Esteban Secaida y Fabi\'an Bustos \end{document}
