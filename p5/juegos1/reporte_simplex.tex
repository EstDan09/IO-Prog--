\documentclass[11pt]{article}
\usepackage[spanish]{babel}
\usepackage[utf8]{inputenc}
\usepackage[T1]{fontenc}
\usepackage[a4paper,margin=2.2cm]{geometry}
\usepackage{amsmath,amssymb,booktabs,array,colortbl}
\usepackage[table,xcdraw]{xcolor}
\usepackage{hyperref,graphicx}
\begin{document}
\begin{titlepage}
\centering
{\Huge Proyecto 4 - Otro S\'implex M\'as}\\[1em]
{\Large juegos1}\\[2em]
{\large Curso: Investigaci\'on de Operaciones\\Semestre: 2025-I}\\[4em]
\vfill
\textbf{Esteban Secaida - Fabian Bustos}\\[2em]
Fecha: \today
\end{titlepage}
\newpage
\section*{Planteamiento del Problema}
Minimizar \[ Z = 1,000x1 +1,000x2 +1,000x3 \]
Sujeto a:\[ 
6,000x1 +2,000x2 +5,000x3 \ge 1,000\\
1,000x1 +7,000x2 +5,000x3 \ge 1,000\\
7,000x1 +5,000x2 +9,000x3 \ge 1,000\\
x_i \ge 0 \text{ para todo } i.\]
\section*{Descripci\'on del M\'etodo S\'implex}
El algoritmo S\'implex, propuesto por George Dantzig en 1947, es un procedimiento iterativo que explora los v\'ertices del poliedro factible para encontrar la soluci\'on \textit{\'optima} de un problema lineal. En cada iteraci\'on se determina una variable que entra a la base y otra que sale, hasta que no existen mejoras posibles en la funci\'on objetivo.\\[1em]
\section*{Tablas del M\'etodo S\'implex}
\begin{table}[h]
\centering
\caption{Tabla inicial.}
\setlength{\tabcolsep}{6pt}
\renewcommand{\arraystretch}{1.15}
\begin{tabular}{lrrrrrrr}
\toprule
 & $x_{1}$ & $x_{2}$ & $x_{3}$ & $s_{1}$ & $s_{2}$ & $s_{3}$ & $b$ \\
\midrule
$Z$ & -13999999,000000 & -13999999,000000 & -18999999,000000 & 1000000,000000 & 1000000,000000 & 1000000,000000 & 0,000000 & 0,000000 & 0,000000 & -3000000,000000 \\
$R_{1}$ & 6,000000 & 2,000000 & 5,000000 & -1,000000 & 0,000000 & 0,000000 & 1,000000 & 0,000000 & 0,000000 & 1,000000 \\
$R_{2}$ & 1,000000 & 7,000000 & 5,000000 & 0,000000 & -1,000000 & 0,000000 & 0,000000 & 1,000000 & 0,000000 & 1,000000 \\
$R_{3}$ & 7,000000 & 5,000000 & 9,000000 & 0,000000 & 0,000000 & -1,000000 & 0,000000 & 0,000000 & 1,000000 & 1,000000 \\
\bottomrule
\end{tabular}
\end{table}

\begin{table}[h]
\centering
\caption{Iteraci\'on 1: entra la columna $x_{3}$ y sale la fila $R_{3}$.}
\setlength{\tabcolsep}{6pt}
\renewcommand{\arraystretch}{1.15}
\begin{tabular}{lrrrrrrr}
\toprule
 & $x_{1}$ & $x_{2}$ & $x_{3}$ & $s_{1}$ & $s_{2}$ & $s_{3}$ & $b$ \\
\midrule
$Z$ & -13999999,000000 & -13999999,000000 & \cellcolor{blue!12}-18999999,000000 & 1000000,000000 & 1000000,000000 & 1000000,000000 & 0,000000 & 0,000000 & 0,000000 & -3000000,000000 \\
$R_{1}$ & 6,000000 & 2,000000 & \cellcolor{blue!12}5,000000 & -1,000000 & 0,000000 & 0,000000 & 1,000000 & 0,000000 & 0,000000 & 1,000000 \\
$R_{2}$ & 1,000000 & 7,000000 & \cellcolor{blue!12}5,000000 & 0,000000 & -1,000000 & 0,000000 & 0,000000 & 1,000000 & 0,000000 & 1,000000 \\
$R_{3}$ & \cellcolor{green!12}7,000000 & \cellcolor{green!12}5,000000 & \cellcolor{orange!35}9,000000 & \cellcolor{green!12}0,000000 & \cellcolor{green!12}0,000000 & \cellcolor{green!12}-1,000000 & \cellcolor{green!12}0,000000 & \cellcolor{green!12}0,000000 & \cellcolor{green!12}1,000000 & \cellcolor{green!12}1,000000 \\
\bottomrule
\end{tabular}
\\[4pt]\textbf{Fracciones } $b_i/a_{i,j}$ para la columna $x_{3}$:\\
$R_{1} = 0,200000$,\ $R_{2} = 0,200000$,\ $R_{3} = 0,111111$ \;\;\textbf{(m\'inima)}.
\end{table}

\begin{table}[h]
\centering
\caption{Iteraci\'on 2: entra la columna $x_{2}$ y sale la fila $R_{2}$.}
\setlength{\tabcolsep}{6pt}
\renewcommand{\arraystretch}{1.15}
\begin{tabular}{lrrrrrrr}
\toprule
 & $x_{1}$ & $x_{2}$ & $x_{3}$ & $s_{1}$ & $s_{2}$ & $s_{3}$ & $b$ \\
\midrule
$Z$ & 777778,000000 & \cellcolor{blue!12}-3444444,000000 & 0,000000 & 1000000,000000 & 1000000,000000 & -1111111,000000 & 0,000000 & 0,000000 & 2111111,000000 & -888889,000000 \\
$R_{1}$ & 2,111111 & \cellcolor{blue!12}-0,777778 & 0,000000 & -1,000000 & 0,000000 & 0,555556 & 1,000000 & 0,000000 & -0,555556 & 0,444444 \\
$R_{2}$ & \cellcolor{green!12}-2,888889 & \cellcolor{orange!35}4,222222 & \cellcolor{green!12}0,000000 & \cellcolor{green!12}0,000000 & \cellcolor{green!12}-1,000000 & \cellcolor{green!12}0,555556 & \cellcolor{green!12}0,000000 & \cellcolor{green!12}1,000000 & \cellcolor{green!12}-0,555556 & \cellcolor{green!12}0,444444 \\
$R_{3}$ & 0,777778 & \cellcolor{blue!12}0,555556 & 1,000000 & 0,000000 & 0,000000 & -0,111111 & 0,000000 & 0,000000 & 0,111111 & 0,111111 \\
\bottomrule
\end{tabular}
\\[4pt]\textbf{Fracciones } $b_i/a_{i,j}$ para la columna $x_{2}$:\\
$R_{2} = 0,105263$ \;\;\textbf{(m\'inima)},\ $R_{3} = 0,200000$.
\end{table}

\begin{table}[h]
\centering
\caption{Iteraci\'on 3: entra la columna $x_{1}$ y sale la fila $R_{3}$.}
\setlength{\tabcolsep}{6pt}
\renewcommand{\arraystretch}{1.15}
\begin{tabular}{lrrrrrrr}
\toprule
 & $x_{1}$ & $x_{2}$ & $x_{3}$ & $s_{1}$ & $s_{2}$ & $s_{3}$ & $b$ \\
\midrule
$Z$ & \cellcolor{blue!12}-1578946,842105 & 0,000000 & 0,000000 & 1000000,000000 & 184210,631579 & -657894,684211 & 0,000000 & 815789,368421 & 1657894,684211 & -526315,947368 \\
$R_{1}$ & \cellcolor{blue!12}1,578947 & 0,000000 & 0,000000 & -1,000000 & -0,184211 & 0,657895 & 1,000000 & 0,184211 & -0,657895 & 0,526316 \\
$R_{2}$ & \cellcolor{blue!12}-0,684211 & 1,000000 & 0,000000 & 0,000000 & -0,236842 & 0,131579 & 0,000000 & 0,236842 & -0,131579 & 0,105263 \\
$R_{3}$ & \cellcolor{orange!35}1,157895 & \cellcolor{green!12}0,000000 & \cellcolor{green!12}1,000000 & \cellcolor{green!12}0,000000 & \cellcolor{green!12}0,131579 & \cellcolor{green!12}-0,184211 & \cellcolor{green!12}0,000000 & \cellcolor{green!12}-0,131579 & \cellcolor{green!12}0,184211 & \cellcolor{green!12}0,052632 \\
\bottomrule
\end{tabular}
\\[4pt]\textbf{Fracciones } $b_i/a_{i,j}$ para la columna $x_{1}$:\\
$R_{1} = 0,333333$,\ $R_{3} = 0,045455$ \;\;\textbf{(m\'inima)}.
\end{table}

\begin{table}[h]
\centering
\caption{Iteraci\'on 4: entra la columna $s_{3}$ y sale la fila $R_{1}$.}
\setlength{\tabcolsep}{6pt}
\renewcommand{\arraystretch}{1.15}
\begin{tabular}{lrrrrrrr}
\toprule
 & $x_{1}$ & $x_{2}$ & $x_{3}$ & $s_{1}$ & $s_{2}$ & $s_{3}$ & $b$ \\
\midrule
$Z$ & 0,000000 & 0,000000 & 1363635,909091 & 1000000,000000 & 363636,409091 & \cellcolor{blue!12}-909090,772727 & 0,000000 & 636363,590909 & 1909090,772727 & -454545,636364 \\
$R_{1}$ & \cellcolor{green!12}0,000000 & \cellcolor{green!12}0,000000 & \cellcolor{green!12}-1,363636 & \cellcolor{green!12}-1,000000 & \cellcolor{green!12}-0,363636 & \cellcolor{orange!35}0,909091 & \cellcolor{green!12}1,000000 & \cellcolor{green!12}0,363636 & \cellcolor{green!12}-0,909091 & \cellcolor{green!12}0,454545 \\
$R_{2}$ & 0,000000 & 1,000000 & 0,590909 & 0,000000 & -0,159091 & \cellcolor{blue!12}0,022727 & 0,000000 & 0,159091 & -0,022727 & 0,136364 \\
$R_{3}$ & 1,000000 & 0,000000 & 0,863636 & 0,000000 & 0,113636 & \cellcolor{blue!12}-0,159091 & 0,000000 & -0,113636 & 0,159091 & 0,045455 \\
\bottomrule
\end{tabular}
\\[4pt]\textbf{Fracciones } $b_i/a_{i,j}$ para la columna $s_{3}$:\\
$R_{1} = 0,500000$ \;\;\textbf{(m\'inima)},\ $R_{2} = 6,000000$,\ .
\end{table}

\begin{table}[h]
\centering
\caption{Iteraci\'on 5: entra la columna $x_{3}$ y sale la fila $R_{3}$.}
\setlength{\tabcolsep}{6pt}
\renewcommand{\arraystretch}{1.15}
\begin{tabular}{lrrrrrrr}
\toprule
 & $x_{1}$ & $x_{2}$ & $x_{3}$ & $s_{1}$ & $s_{2}$ & $s_{3}$ & $b$ \\
\midrule
$Z$ & 0,000000 & 0,000000 & \cellcolor{blue!12}-0,250000 & 0,150000 & 0,100000 & 0,000000 & 999999,850000 & 999999,900000 & 1000000,000000 & -0,250000 \\
$R_{1}$ & 0,000000 & 0,000000 & \cellcolor{blue!12}-1,500000 & -1,100000 & -0,400000 & 1,000000 & 1,100000 & 0,400000 & -1,000000 & 0,500000 \\
$R_{2}$ & 0,000000 & 1,000000 & \cellcolor{blue!12}0,625000 & 0,025000 & -0,150000 & 0,000000 & -0,025000 & 0,150000 & 0,000000 & 0,125000 \\
$R_{3}$ & \cellcolor{green!12}1,000000 & \cellcolor{green!12}0,000000 & \cellcolor{orange!35}0,625000 & \cellcolor{green!12}-0,175000 & \cellcolor{green!12}0,050000 & \cellcolor{green!12}0,000000 & \cellcolor{green!12}0,175000 & \cellcolor{green!12}-0,050000 & \cellcolor{green!12}0,000000 & \cellcolor{green!12}0,125000 \\
\bottomrule
\end{tabular}
\\[4pt]\textbf{Fracciones } $b_i/a_{i,j}$ para la columna $x_{3}$:\\
$R_{2} = 0,200000$,\ $R_{3} = 0,200000$ \;\;\textbf{(m\'inima)}.
\end{table}

\begin{table}[h]
\centering
\caption{Tabla final.}
\setlength{\tabcolsep}{6pt}
\renewcommand{\arraystretch}{1.15}
\begin{tabular}{lrrrrrrr}
\toprule
 & $x_{1}$ & $x_{2}$ & $x_{3}$ & $s_{1}$ & $s_{2}$ & $s_{3}$ & $b$ \\
\midrule
$Z$ & 0,400000 & 0,000000 & 0,000000 & 0,080000 & 0,120000 & 0,000000 & 999999,920000 & 999999,880000 & 1000000,000000 & -0,200000 \\
$R_{1}$ & 2,400000 & 0,000000 & 0,000000 & -1,520000 & -0,280000 & 1,000000 & 1,520000 & 0,280000 & -1,000000 & 0,800000 \\
$R_{2}$ & -1,000000 & 1,000000 & 0,000000 & 0,200000 & -0,200000 & 0,000000 & -0,200000 & 0,200000 & 0,000000 & 0,000000 \\
$R_{3}$ & 1,600000 & 0,000000 & 1,000000 & -0,280000 & 0,080000 & 0,000000 & 0,280000 & -0,080000 & 0,000000 & 0,200000 \\
\bottomrule
\end{tabular}
\end{table}

\section*{Resultados y Casos Especiales}
Estado del problema: \textbf{Óptimo}.\\
Valor \textit{\'optimo}: $Z^* = 0,200000$.\\[4pt]
Soluci\'on \textit{\'optima}:\\[4pt]
\[ x1 = 0,000000,\;x2 = 0,000000,\;x3 = 0,200000. \]
\vfill\smallskip\noindent Documento generado por Otro Simplex mas, de Esteban Secaida y Fabi\'an Bustos \end{document}
