\documentclass[11pt]{article}
\usepackage[spanish]{babel}
\usepackage[utf8]{inputenc}
\usepackage[T1]{fontenc}
\usepackage[a4paper,margin=2.2cm]{geometry}
\usepackage{amsmath,amssymb,booktabs,array,colortbl}
\usepackage[table,xcdraw]{xcolor}
\usepackage{hyperref,graphicx}
\begin{document}
\begin{titlepage}
\centering
{\Huge Proyecto 4 - Otro S\'implex M\'as}\\[1em]
{\Large juegos2}\\[2em]
{\large Curso: Investigaci\'on de Operaciones\\Semestre: 2025-I}\\[4em]
\vfill
\textbf{Esteban Secaida - Fabian Bustos}\\[2em]
Fecha: \today
\end{titlepage}
\newpage
\section*{Planteamiento del Problema}
Minimizar \[ Z = 1,000x1 +1,000x2 +1,000x3 \]
Sujeto a:\[ 
3,000x1 +1,000x2 +1,000x3 \ge 1,000\\
1,000x1 +1,000x2 +4,000x3 \ge 1,000\\
1,000x1 +5,000x2 +1,000x3 \ge 1,000\\
x_i \ge 0 \text{ para todo } i.\]
\section*{Descripci\'on del M\'etodo S\'implex}
El algoritmo S\'implex, propuesto por George Dantzig en 1947, es un procedimiento iterativo que explora los v\'ertices del poliedro factible para encontrar la soluci\'on \textit{\'optima} de un problema lineal. En cada iteraci\'on se determina una variable que entra a la base y otra que sale, hasta que no existen mejoras posibles en la funci\'on objetivo.\\[1em]
\section*{Tablas del M\'etodo S\'implex}
\begin{table}[h]
\centering
\caption{Tabla inicial.}
\setlength{\tabcolsep}{6pt}
\renewcommand{\arraystretch}{1.15}
\begin{tabular}{lrrrrrrr}
\toprule
 & $x_{1}$ & $x_{2}$ & $x_{3}$ & $s_{1}$ & $s_{2}$ & $s_{3}$ & $b$ \\
\midrule
$Z$ & -4999999,000000 & -6999999,000000 & -5999999,000000 & 1000000,000000 & 1000000,000000 & 1000000,000000 & 0,000000 & 0,000000 & 0,000000 & -3000000,000000 \\
$R_{1}$ & 3,000000 & 1,000000 & 1,000000 & -1,000000 & 0,000000 & 0,000000 & 1,000000 & 0,000000 & 0,000000 & 1,000000 \\
$R_{2}$ & 1,000000 & 1,000000 & 4,000000 & 0,000000 & -1,000000 & 0,000000 & 0,000000 & 1,000000 & 0,000000 & 1,000000 \\
$R_{3}$ & 1,000000 & 5,000000 & 1,000000 & 0,000000 & 0,000000 & -1,000000 & 0,000000 & 0,000000 & 1,000000 & 1,000000 \\
\bottomrule
\end{tabular}
\end{table}

\begin{table}[h]
\centering
\caption{Iteraci\'on 1: entra la columna $x_{2}$ y sale la fila $R_{3}$.}
\setlength{\tabcolsep}{6pt}
\renewcommand{\arraystretch}{1.15}
\begin{tabular}{lrrrrrrr}
\toprule
 & $x_{1}$ & $x_{2}$ & $x_{3}$ & $s_{1}$ & $s_{2}$ & $s_{3}$ & $b$ \\
\midrule
$Z$ & -4999999,000000 & \cellcolor{blue!12}-6999999,000000 & -5999999,000000 & 1000000,000000 & 1000000,000000 & 1000000,000000 & 0,000000 & 0,000000 & 0,000000 & -3000000,000000 \\
$R_{1}$ & 3,000000 & \cellcolor{blue!12}1,000000 & 1,000000 & -1,000000 & 0,000000 & 0,000000 & 1,000000 & 0,000000 & 0,000000 & 1,000000 \\
$R_{2}$ & 1,000000 & \cellcolor{blue!12}1,000000 & 4,000000 & 0,000000 & -1,000000 & 0,000000 & 0,000000 & 1,000000 & 0,000000 & 1,000000 \\
$R_{3}$ & \cellcolor{green!12}1,000000 & \cellcolor{orange!35}5,000000 & \cellcolor{green!12}1,000000 & \cellcolor{green!12}0,000000 & \cellcolor{green!12}0,000000 & \cellcolor{green!12}-1,000000 & \cellcolor{green!12}0,000000 & \cellcolor{green!12}0,000000 & \cellcolor{green!12}1,000000 & \cellcolor{green!12}1,000000 \\
\bottomrule
\end{tabular}
\\[4pt]\textbf{Fracciones } $b_i/a_{i,j}$ para la columna $x_{2}$:\\
$R_{1} = 1,000000$,\ $R_{2} = 1,000000$,\ $R_{3} = 0,200000$ \;\;\textbf{(m\'inima)}.
\end{table}

\begin{table}[h]
\centering
\caption{Iteraci\'on 2: entra la columna $x_{3}$ y sale la fila $R_{2}$.}
\setlength{\tabcolsep}{6pt}
\renewcommand{\arraystretch}{1.15}
\begin{tabular}{lrrrrrrr}
\toprule
 & $x_{1}$ & $x_{2}$ & $x_{3}$ & $s_{1}$ & $s_{2}$ & $s_{3}$ & $b$ \\
\midrule
$Z$ & -3599999,200000 & 0,000000 & \cellcolor{blue!12}-4599999,200000 & 1000000,000000 & 1000000,000000 & -399999,800000 & 0,000000 & 0,000000 & 1399999,800000 & -1600000,200000 \\
$R_{1}$ & 2,800000 & 0,000000 & \cellcolor{blue!12}0,800000 & -1,000000 & 0,000000 & 0,200000 & 1,000000 & 0,000000 & -0,200000 & 0,800000 \\
$R_{2}$ & \cellcolor{green!12}0,800000 & \cellcolor{green!12}0,000000 & \cellcolor{orange!35}3,800000 & \cellcolor{green!12}0,000000 & \cellcolor{green!12}-1,000000 & \cellcolor{green!12}0,200000 & \cellcolor{green!12}0,000000 & \cellcolor{green!12}1,000000 & \cellcolor{green!12}-0,200000 & \cellcolor{green!12}0,800000 \\
$R_{3}$ & 0,200000 & 1,000000 & \cellcolor{blue!12}0,200000 & 0,000000 & 0,000000 & -0,200000 & 0,000000 & 0,000000 & 0,200000 & 0,200000 \\
\bottomrule
\end{tabular}
\\[4pt]\textbf{Fracciones } $b_i/a_{i,j}$ para la columna $x_{3}$:\\
$R_{1} = 1,000000$,\ $R_{2} = 0,210526$ \;\;\textbf{(m\'inima)},\ $R_{3} = 1,000000$.
\end{table}

\begin{table}[h]
\centering
\caption{Iteraci\'on 3: entra la columna $x_{1}$ y sale la fila $R_{1}$.}
\setlength{\tabcolsep}{6pt}
\renewcommand{\arraystretch}{1.15}
\begin{tabular}{lrrrrrrr}
\toprule
 & $x_{1}$ & $x_{2}$ & $x_{3}$ & $s_{1}$ & $s_{2}$ & $s_{3}$ & $b$ \\
\midrule
$Z$ & \cellcolor{blue!12}-2631578,315789 & 0,000000 & 0,000000 & 1000000,000000 & -210526,105263 & -157894,578947 & 0,000000 & 1210526,105263 & 1157894,578947 & -631579,315789 \\
$R_{1}$ & \cellcolor{orange!35}2,631579 & \cellcolor{green!12}0,000000 & \cellcolor{green!12}0,000000 & \cellcolor{green!12}-1,000000 & \cellcolor{green!12}0,210526 & \cellcolor{green!12}0,157895 & \cellcolor{green!12}1,000000 & \cellcolor{green!12}-0,210526 & \cellcolor{green!12}-0,157895 & \cellcolor{green!12}0,631579 \\
$R_{2}$ & \cellcolor{blue!12}0,210526 & 0,000000 & 1,000000 & 0,000000 & -0,263158 & 0,052632 & 0,000000 & 0,263158 & -0,052632 & 0,210526 \\
$R_{3}$ & \cellcolor{blue!12}0,157895 & 1,000000 & 0,000000 & 0,000000 & 0,052632 & -0,210526 & 0,000000 & -0,052632 & 0,210526 & 0,157895 \\
\bottomrule
\end{tabular}
\\[4pt]\textbf{Fracciones } $b_i/a_{i,j}$ para la columna $x_{1}$:\\
$R_{1} = 0,240000$ \;\;\textbf{(m\'inima)},\ $R_{2} = 1,000000$,\ $R_{3} = 1,000000$.
\end{table}

\begin{table}[h]
\centering
\caption{Tabla final.}
\setlength{\tabcolsep}{6pt}
\renewcommand{\arraystretch}{1.15}
\begin{tabular}{lrrrrrrr}
\toprule
 & $x_{1}$ & $x_{2}$ & $x_{3}$ & $s_{1}$ & $s_{2}$ & $s_{3}$ & $b$ \\
\midrule
$Z$ & 0,000000 & 0,000000 & 0,000000 & 0,240000 & 0,160000 & 0,120000 & 999999,760000 & 999999,840000 & 999999,880000 & -0,520000 \\
$R_{1}$ & 1,000000 & 0,000000 & 0,000000 & -0,380000 & 0,080000 & 0,060000 & 0,380000 & -0,080000 & -0,060000 & 0,240000 \\
$R_{2}$ & 0,000000 & 0,000000 & 1,000000 & 0,080000 & -0,280000 & 0,040000 & -0,080000 & 0,280000 & -0,040000 & 0,160000 \\
$R_{3}$ & 0,000000 & 1,000000 & 0,000000 & 0,060000 & 0,040000 & -0,220000 & -0,060000 & -0,040000 & 0,220000 & 0,120000 \\
\bottomrule
\end{tabular}
\end{table}

\section*{Resultados y Casos Especiales}
Estado del problema: \textbf{Óptimo}.\\
Valor \textit{\'optimo}: $Z^* = 0,520000$.\\[4pt]
Soluci\'on \textit{\'optima}:\\[4pt]
\[ x1 = 0,240000,\;x2 = 0,120000,\;x3 = 0,160000. \]
\vfill\smallskip\noindent Documento generado por Otro Simplex mas, de Esteban Secaida y Fabi\'an Bustos \end{document}
