\documentclass[11pt]{article}
\usepackage[spanish]{babel}
\usepackage[utf8]{inputenc}
\usepackage[T1]{fontenc}
\usepackage[a4paper,margin=2.2cm]{geometry}
\usepackage{amsmath,amssymb,booktabs,array,colortbl}
\usepackage[table,xcdraw]{xcolor}
\usepackage{hyperref,graphicx}
\begin{document}
\begin{titlepage}
\centering
{\Huge Proyecto 4 - Otro S\'implex M\'as}\\[1em]
{\Large juegos3}\\[2em]
{\large Curso: Investigaci\'on de Operaciones\\Semestre: 2025-I}\\[4em]
\vfill
\textbf{Esteban Secaida - Fabian Bustos}\\[2em]
Fecha: \today
\end{titlepage}
\newpage
\section*{Planteamiento del Problema}
Minimizar \[ Z = 1,000x1 +1,000x2 +1,000x3 +1,000x4 \]
Sujeto a:\[ 
5,000x1 +3,000x2 +8,000x3 +5,000x4 \ge 1,000\\
7,000x1 +5,000x2 +5,000x3 +2,000x4 \ge 1,000\\
2,000x1 +5,000x2 +5,000x3 +9,000x4 \ge 1,000\\
5,000x1 +8,000x2 +1,000x3 +5,000x4 \ge 1,000\\
x_i \ge 0 \text{ para todo } i.\]
\section*{Descripci\'on del M\'etodo S\'implex}
El algoritmo S\'implex, propuesto por George Dantzig en 1947, es un procedimiento iterativo que explora los v\'ertices del poliedro factible para encontrar la soluci\'on \textit{\'optima} de un problema lineal. En cada iteraci\'on se determina una variable que entra a la base y otra que sale, hasta que no existen mejoras posibles en la funci\'on objetivo.\\[1em]
\section*{Tablas del M\'etodo S\'implex}
\begin{table}[h]
\centering
\caption{Tabla inicial.}
\setlength{\tabcolsep}{6pt}
\renewcommand{\arraystretch}{1.15}
\begin{tabular}{lrrrrrrrrr}
\toprule
 & $x_{1}$ & $x_{2}$ & $x_{3}$ & $x_{4}$ & $s_{1}$ & $s_{2}$ & $s_{3}$ & $s_{4}$ & $b$ \\
\midrule
$Z$ & -18999999,000000 & -20999999,000000 & -18999999,000000 & -20999999,000000 & 1000000,000000 & 1000000,000000 & 1000000,000000 & 1000000,000000 & 0,000000 & 0,000000 & 0,000000 & 0,000000 & -4000000,000000 \\
$R_{1}$ & 5,000000 & 3,000000 & 8,000000 & 5,000000 & -1,000000 & 0,000000 & 0,000000 & 0,000000 & 1,000000 & 0,000000 & 0,000000 & 0,000000 & 1,000000 \\
$R_{2}$ & 7,000000 & 5,000000 & 5,000000 & 2,000000 & 0,000000 & -1,000000 & 0,000000 & 0,000000 & 0,000000 & 1,000000 & 0,000000 & 0,000000 & 1,000000 \\
$R_{3}$ & 2,000000 & 5,000000 & 5,000000 & 9,000000 & 0,000000 & 0,000000 & -1,000000 & 0,000000 & 0,000000 & 0,000000 & 1,000000 & 0,000000 & 1,000000 \\
$R_{4}$ & 5,000000 & 8,000000 & 1,000000 & 5,000000 & 0,000000 & 0,000000 & 0,000000 & -1,000000 & 0,000000 & 0,000000 & 0,000000 & 1,000000 & 1,000000 \\
\bottomrule
\end{tabular}
\end{table}

\begin{table}[h]
\centering
\caption{Iteraci\'on 1: entra la columna $x_{2}$ y sale la fila $R_{4}$.}
\setlength{\tabcolsep}{6pt}
\renewcommand{\arraystretch}{1.15}
\begin{tabular}{lrrrrrrrrr}
\toprule
 & $x_{1}$ & $x_{2}$ & $x_{3}$ & $x_{4}$ & $s_{1}$ & $s_{2}$ & $s_{3}$ & $s_{4}$ & $b$ \\
\midrule
$Z$ & -18999999,000000 & \cellcolor{blue!12}-20999999,000000 & -18999999,000000 & -20999999,000000 & 1000000,000000 & 1000000,000000 & 1000000,000000 & 1000000,000000 & 0,000000 & 0,000000 & 0,000000 & 0,000000 & -4000000,000000 \\
$R_{1}$ & 5,000000 & \cellcolor{blue!12}3,000000 & 8,000000 & 5,000000 & -1,000000 & 0,000000 & 0,000000 & 0,000000 & 1,000000 & 0,000000 & 0,000000 & 0,000000 & 1,000000 \\
$R_{2}$ & 7,000000 & \cellcolor{blue!12}5,000000 & 5,000000 & 2,000000 & 0,000000 & -1,000000 & 0,000000 & 0,000000 & 0,000000 & 1,000000 & 0,000000 & 0,000000 & 1,000000 \\
$R_{3}$ & 2,000000 & \cellcolor{blue!12}5,000000 & 5,000000 & 9,000000 & 0,000000 & 0,000000 & -1,000000 & 0,000000 & 0,000000 & 0,000000 & 1,000000 & 0,000000 & 1,000000 \\
$R_{4}$ & \cellcolor{green!12}5,000000 & \cellcolor{orange!35}8,000000 & \cellcolor{green!12}1,000000 & \cellcolor{green!12}5,000000 & \cellcolor{green!12}0,000000 & \cellcolor{green!12}0,000000 & \cellcolor{green!12}0,000000 & \cellcolor{green!12}-1,000000 & \cellcolor{green!12}0,000000 & \cellcolor{green!12}0,000000 & \cellcolor{green!12}0,000000 & \cellcolor{green!12}1,000000 & \cellcolor{green!12}1,000000 \\
\bottomrule
\end{tabular}
\\[4pt]\textbf{Fracciones } $b_i/a_{i,j}$ para la columna $x_{2}$:\\
$R_{1} = 0,333333$,\ $R_{2} = 0,200000$,\ $R_{3} = 0,200000$,\ $R_{4} = 0,125000$ \;\;\textbf{(m\'inima)}.
\end{table}

\begin{table}[h]
\centering
\caption{Iteraci\'on 2: entra la columna $x_{3}$ y sale la fila $R_{1}$.}
\setlength{\tabcolsep}{6pt}
\renewcommand{\arraystretch}{1.15}
\begin{tabular}{lrrrrrrrrr}
\toprule
 & $x_{1}$ & $x_{2}$ & $x_{3}$ & $x_{4}$ & $s_{1}$ & $s_{2}$ & $s_{3}$ & $s_{4}$ & $b$ \\
\midrule
$Z$ & -5874999,625000 & 0,000000 & \cellcolor{blue!12}-16374999,125000 & -7874999,625000 & 1000000,000000 & 1000000,000000 & 1000000,000000 & -1624999,875000 & 0,000000 & 0,000000 & 0,000000 & 2624999,875000 & -1375000,125000 \\
$R_{1}$ & \cellcolor{green!12}3,125000 & \cellcolor{green!12}0,000000 & \cellcolor{orange!35}7,625000 & \cellcolor{green!12}3,125000 & \cellcolor{green!12}-1,000000 & \cellcolor{green!12}0,000000 & \cellcolor{green!12}0,000000 & \cellcolor{green!12}0,375000 & \cellcolor{green!12}1,000000 & \cellcolor{green!12}0,000000 & \cellcolor{green!12}0,000000 & \cellcolor{green!12}-0,375000 & \cellcolor{green!12}0,625000 \\
$R_{2}$ & 3,875000 & 0,000000 & \cellcolor{blue!12}4,375000 & -1,125000 & 0,000000 & -1,000000 & 0,000000 & 0,625000 & 0,000000 & 1,000000 & 0,000000 & -0,625000 & 0,375000 \\
$R_{3}$ & -1,125000 & 0,000000 & \cellcolor{blue!12}4,375000 & 5,875000 & 0,000000 & 0,000000 & -1,000000 & 0,625000 & 0,000000 & 0,000000 & 1,000000 & -0,625000 & 0,375000 \\
$R_{4}$ & 0,625000 & 1,000000 & \cellcolor{blue!12}0,125000 & 0,625000 & 0,000000 & 0,000000 & 0,000000 & -0,125000 & 0,000000 & 0,000000 & 0,000000 & 0,125000 & 0,125000 \\
\bottomrule
\end{tabular}
\\[4pt]\textbf{Fracciones } $b_i/a_{i,j}$ para la columna $x_{3}$:\\
$R_{1} = 0,081967$ \;\;\textbf{(m\'inima)},\ $R_{2} = 0,085714$,\ $R_{3} = 0,085714$,\ $R_{4} = 1,000000$.
\end{table}

\begin{table}[h]
\centering
\caption{Iteraci\'on 3: entra la columna $x_{4}$ y sale la fila $R_{3}$.}
\setlength{\tabcolsep}{6pt}
\renewcommand{\arraystretch}{1.15}
\begin{tabular}{lrrrrrrrrr}
\toprule
 & $x_{1}$ & $x_{2}$ & $x_{3}$ & $x_{4}$ & $s_{1}$ & $s_{2}$ & $s_{3}$ & $s_{4}$ & $b$ \\
\midrule
$Z$ & 836065,590164 & 0,000000 & 0,000000 & \cellcolor{blue!12}-1163934,409836 & -1147540,868852 & 1000000,000000 & 1000000,000000 & -819672,049180 & 2147540,868852 & 0,000000 & 0,000000 & 1819672,049180 & -32787,081967 \\
$R_{1}$ & 0,409836 & 0,000000 & 1,000000 & \cellcolor{blue!12}0,409836 & -0,131148 & 0,000000 & 0,000000 & 0,049180 & 0,131148 & 0,000000 & 0,000000 & -0,049180 & 0,081967 \\
$R_{2}$ & 2,081967 & 0,000000 & 0,000000 & \cellcolor{blue!12}-2,918033 & 0,573770 & -1,000000 & 0,000000 & 0,409836 & -0,573770 & 1,000000 & 0,000000 & -0,409836 & 0,016393 \\
$R_{3}$ & \cellcolor{green!12}-2,918033 & \cellcolor{green!12}0,000000 & \cellcolor{green!12}0,000000 & \cellcolor{orange!35}4,081967 & \cellcolor{green!12}0,573770 & \cellcolor{green!12}0,000000 & \cellcolor{green!12}-1,000000 & \cellcolor{green!12}0,409836 & \cellcolor{green!12}-0,573770 & \cellcolor{green!12}0,000000 & \cellcolor{green!12}1,000000 & \cellcolor{green!12}-0,409836 & \cellcolor{green!12}0,016393 \\
$R_{4}$ & 0,573770 & 1,000000 & 0,000000 & \cellcolor{blue!12}0,573770 & 0,016393 & 0,000000 & 0,000000 & -0,131148 & -0,016393 & 0,000000 & 0,000000 & 0,131148 & 0,114754 \\
\bottomrule
\end{tabular}
\\[4pt]\textbf{Fracciones } $b_i/a_{i,j}$ para la columna $x_{4}$:\\
$R_{1} = 0,200000$,\ $R_{3} = 0,004016$ \;\;\textbf{(m\'inima)},\ $R_{4} = 0,200000$.
\end{table}

\begin{table}[h]
\centering
\caption{Iteraci\'on 4: entra la columna $s_{1}$ y sale la fila $R_{3}$.}
\setlength{\tabcolsep}{6pt}
\renewcommand{\arraystretch}{1.15}
\begin{tabular}{lrrrrrrrrr}
\toprule
 & $x_{1}$ & $x_{2}$ & $x_{3}$ & $x_{4}$ & $s_{1}$ & $s_{2}$ & $s_{3}$ & $s_{4}$ & $b$ \\
\midrule
$Z$ & 4016,092369 & 0,000000 & 0,000000 & 0,000000 & \cellcolor{blue!12}-983935,630522 & 1000000,000000 & 714859,441767 & -702811,164659 & 1983935,630522 & 0,000000 & 285140,558233 & 1702811,164659 & -28112,646586 \\
$R_{1}$ & 0,702811 & 0,000000 & 1,000000 & 0,000000 & \cellcolor{blue!12}-0,188755 & 0,000000 & 0,100402 & 0,008032 & 0,188755 & 0,000000 & -0,100402 & -0,008032 & 0,080321 \\
$R_{2}$ & -0,004016 & 0,000000 & 0,000000 & 0,000000 & \cellcolor{blue!12}0,983936 & -1,000000 & -0,714859 & 0,702811 & -0,983936 & 1,000000 & 0,714859 & -0,702811 & 0,028112 \\
$R_{3}$ & \cellcolor{green!12}-0,714859 & \cellcolor{green!12}0,000000 & \cellcolor{green!12}0,000000 & \cellcolor{green!12}1,000000 & \cellcolor{orange!35}0,140562 & \cellcolor{green!12}0,000000 & \cellcolor{green!12}-0,244980 & \cellcolor{green!12}0,100402 & \cellcolor{green!12}-0,140562 & \cellcolor{green!12}0,000000 & \cellcolor{green!12}0,244980 & \cellcolor{green!12}-0,100402 & \cellcolor{green!12}0,004016 \\
$R_{4}$ & 0,983936 & 1,000000 & 0,000000 & 0,000000 & \cellcolor{blue!12}-0,064257 & 0,000000 & 0,140562 & -0,188755 & 0,064257 & 0,000000 & -0,140562 & 0,188755 & 0,112450 \\
\bottomrule
\end{tabular}
\\[4pt]\textbf{Fracciones } $b_i/a_{i,j}$ para la columna $s_{1}$:\\
$R_{2} = 0,028571$,\ $R_{3} = 0,028571$ \;\;\textbf{(m\'inima)},\ .
\end{table}

\begin{table}[h]
\centering
\caption{Iteraci\'on 5: entra la columna $x_{1}$ y sale la fila $R_{2}$.}
\setlength{\tabcolsep}{6pt}
\renewcommand{\arraystretch}{1.15}
\begin{tabular}{lrrrrrrrrr}
\toprule
 & $x_{1}$ & $x_{2}$ & $x_{3}$ & $x_{4}$ & $s_{1}$ & $s_{2}$ & $s_{3}$ & $s_{4}$ & $b$ \\
\midrule
$Z$ & \cellcolor{blue!12}-4999999,400000 & 0,000000 & 0,000000 & 6999999,200000 & 0,000000 & 1000000,000000 & -999999,800000 & 0,000000 & 1000000,000000 & 0,000000 & 1999999,800000 & 1000000,000000 & -0,200000 \\
$R_{1}$ & \cellcolor{blue!12}-0,257143 & 0,000000 & 1,000000 & 1,342857 & 0,000000 & 0,000000 & -0,228571 & 0,142857 & 0,000000 & 0,000000 & 0,228571 & -0,142857 & 0,085714 \\
$R_{2}$ & \cellcolor{orange!35}5,000000 & \cellcolor{green!12}0,000000 & \cellcolor{green!12}0,000000 & \cellcolor{green!12}-7,000000 & \cellcolor{green!12}0,000000 & \cellcolor{green!12}-1,000000 & \cellcolor{green!12}1,000000 & \cellcolor{green!12}0,000000 & \cellcolor{green!12}0,000000 & \cellcolor{green!12}1,000000 & \cellcolor{green!12}-1,000000 & \cellcolor{green!12}0,000000 & \cellcolor{green!12}0,000000 \\
$R_{3}$ & \cellcolor{blue!12}-5,085714 & 0,000000 & 0,000000 & 7,114286 & 1,000000 & 0,000000 & -1,742857 & 0,714286 & -1,000000 & 0,000000 & 1,742857 & -0,714286 & 0,028571 \\
$R_{4}$ & \cellcolor{blue!12}0,657143 & 1,000000 & 0,000000 & 0,457143 & 0,000000 & 0,000000 & 0,028571 & -0,142857 & 0,000000 & 0,000000 & -0,028571 & 0,142857 & 0,114286 \\
\bottomrule
\end{tabular}
\\[4pt]\textbf{Fracciones } $b_i/a_{i,j}$ para la columna $x_{1}$:\\
$R_{2} = 0,000000$ \;\;\textbf{(m\'inima)},\ $R_{4} = 0,173913$.
\end{table}

\begin{table}[h]
\centering
\caption{Tabla final.}
\setlength{\tabcolsep}{6pt}
\renewcommand{\arraystretch}{1.15}
\begin{tabular}{lrrrrrrrrr}
\toprule
 & $x_{1}$ & $x_{2}$ & $x_{3}$ & $x_{4}$ & $s_{1}$ & $s_{2}$ & $s_{3}$ & $s_{4}$ & $b$ \\
\midrule
$Z$ & 0,000000 & 0,000000 & 0,000000 & 0,040000 & 0,000000 & 0,120000 & 0,080000 & 0,000000 & 1000000,000000 & 999999,880000 & 999999,920000 & 1000000,000000 & -0,200000 \\
$R_{1}$ & 0,000000 & 0,000000 & 1,000000 & 0,982857 & 0,000000 & -0,051429 & -0,177143 & 0,142857 & 0,000000 & 0,051429 & 0,177143 & -0,142857 & 0,085714 \\
$R_{2}$ & 1,000000 & 0,000000 & 0,000000 & -1,400000 & 0,000000 & -0,200000 & 0,200000 & 0,000000 & 0,000000 & 0,200000 & -0,200000 & 0,000000 & 0,000000 \\
$R_{3}$ & 0,000000 & 0,000000 & 0,000000 & -0,005714 & 1,000000 & -1,017143 & -0,725714 & 0,714286 & -1,000000 & 1,017143 & 0,725714 & -0,714286 & 0,028571 \\
$R_{4}$ & 0,000000 & 1,000000 & 0,000000 & 1,377143 & 0,000000 & 0,131429 & -0,102857 & -0,142857 & 0,000000 & -0,131429 & 0,102857 & 0,142857 & 0,114286 \\
\bottomrule
\end{tabular}
\end{table}

\section*{Resultados y Casos Especiales}
Estado del problema: \textbf{Óptimo}.\\
Valor \textit{\'optimo}: $Z^* = 0,200000$.\\[4pt]
Soluci\'on \textit{\'optima}:\\[4pt]
\[ x1 = -0,000000,\;x2 = 0,114286,\;x3 = 0,085714,\;x4 = 0,000000. \]
\emph{Nota:} Se detect\'o degeneraci\'on (al menos un ratio m\'inimo fue 0). Se aplic\'o la regla de Bland para evitar ciclos.\\[6pt]
\vfill\smallskip\noindent Documento generado por Otro Simplex mas, de Esteban Secaida y Fabi\'an Bustos \end{document}
