\documentclass[11pt]{article}
\usepackage[spanish]{babel}
\usepackage[utf8]{inputenc}
\usepackage[T1]{fontenc}
\usepackage[a4paper,margin=2.2cm]{geometry}
\usepackage{amsmath,amssymb,booktabs,array,colortbl}
\usepackage[table,xcdraw]{xcolor}
\usepackage{hyperref,graphicx}
\begin{document}
\begin{titlepage}
\centering
{\Huge Proyecto 4 - Otro S\'implex M\'as}\\[1em]
{\Large problema8}\\[2em]
{\large Curso: Investigaci\'on de Operaciones\\Semestre: 2025-I}\\[4em]
\vfill
\textbf{Esteban Secaida - Fabian Bustos}\\[2em]
Fecha: \today
\end{titlepage}
\newpage
\section*{Planteamiento del Problema}
Maximizar \[ Z = 36,000x_1 +30,000x_2 -3,000x_3 -4,000x_4 \]
Sujeto a:\[ 
1,000x_1 
+1,000x_2 
-1,000x_3 
+0,000x_4 
\le 5,000\\
6,000x_1 
+5,000x_2 
+0,000x_3 
-1,000x_4 
\le 10,000\\
x_i \ge 0 \text{ para todo } i.\]
\section*{Descripci\'on del M\'etodo S\'implex}
El algoritmo S\'implex, propuesto por George Dantzig en 1947, es un procedimiento iterativo que explora los v\'ertices del poliedro factible para encontrar la soluci\'on \textit{\'optima} de un problema lineal. En cada iteraci\'on se determina una variable que entra a la base y otra que sale, hasta que no existen mejoras posibles en la funci\'on objetivo.\\[1em]
\section*{Tablas del M\'etodo S\'implex}
\begin{table}[h]
\centering
\caption{Tabla inicial.}
\setlength{\tabcolsep}{6pt}
\renewcommand{\arraystretch}{1.15}
\begin{tabular}{lrrrrrrr}
\toprule
 & $x_{1}$ & $x_{2}$ & $x_{3}$ & $x_{4}$ & $s_{1}$ & $s_{2}$ & $b$ \\
\midrule
$Z$ & -36,000000 & -30,000000 & 3,000000 & 4,000000 & 0,000000 & 0,000000 & 0,000000 \\
$R_{1}$ & 1,000000 & 1,000000 & -1,000000 & 0,000000 & 1,000000 & 0,000000 & 5,000000 \\
$R_{2}$ & 6,000000 & 5,000000 & 0,000000 & -1,000000 & 0,000000 & 1,000000 & 10,000000 \\
\bottomrule
\end{tabular}
\end{table}

\begin{table}[h]
\centering
\caption{Iteraci\'on 1: entra la columna $x_{1}$ y sale la fila $R_{2}$.}
\setlength{\tabcolsep}{6pt}
\renewcommand{\arraystretch}{1.15}
\begin{tabular}{lrrrrrrr}
\toprule
 & $x_{1}$ & $x_{2}$ & $x_{3}$ & $x_{4}$ & $s_{1}$ & $s_{2}$ & $b$ \\
\midrule
$Z$ & \cellcolor{blue!12}-36,000000 & -30,000000 & 3,000000 & 4,000000 & 0,000000 & 0,000000 & 0,000000 \\
$R_{1}$ & \cellcolor{blue!12}1,000000 & 1,000000 & -1,000000 & 0,000000 & 1,000000 & 0,000000 & 5,000000 \\
$R_{2}$ & \cellcolor{orange!35}6,000000 & \cellcolor{green!12}5,000000 & \cellcolor{green!12}0,000000 & \cellcolor{green!12}-1,000000 & \cellcolor{green!12}0,000000 & \cellcolor{green!12}1,000000 & \cellcolor{green!12}10,000000 \\
\bottomrule
\end{tabular}
\\[4pt]\textbf{Fracciones } $b_i/a_{i,j}$ para la columna $x_{1}$:\\
$R_{1} = 5,000000$,\ $R_{2} = 1,666667$ \;\;\textbf{(m\'inima)}.
\end{table}

\begin{table}[h]
\centering
\caption{Iteraci\'on 2: entra la columna $x_{4}$ y sale la fila $R_{1}$.}
\setlength{\tabcolsep}{6pt}
\renewcommand{\arraystretch}{1.15}
\begin{tabular}{lrrrrrrr}
\toprule
 & $x_{1}$ & $x_{2}$ & $x_{3}$ & $x_{4}$ & $s_{1}$ & $s_{2}$ & $b$ \\
\midrule
$Z$ & 0,000000 & 0,000000 & 3,000000 & \cellcolor{blue!12}-2,000000 & 0,000000 & 6,000000 & 60,000000 \\
$R_{1}$ & \cellcolor{green!12}0,000000 & \cellcolor{green!12}0,166667 & \cellcolor{green!12}-1,000000 & \cellcolor{orange!35}0,166667 & \cellcolor{green!12}1,000000 & \cellcolor{green!12}-0,166667 & \cellcolor{green!12}3,333333 \\
$R_{2}$ & 1,000000 & 0,833333 & 0,000000 & \cellcolor{blue!12}-0,166667 & 0,000000 & 0,166667 & 1,666667 \\
\bottomrule
\end{tabular}
\\[4pt]\textbf{Fracciones } $b_i/a_{i,j}$ para la columna $x_{4}$:\\
$R_{1} = 20,000000$ \;\;\textbf{(m\'inima)},\ .
\end{table}

\begin{table}[h]
\centering
\caption{Tabla final.}
\setlength{\tabcolsep}{6pt}
\renewcommand{\arraystretch}{1.15}
\begin{tabular}{lrrrrrrr}
\toprule
 & $x_{1}$ & $x_{2}$ & $x_{3}$ & $x_{4}$ & $s_{1}$ & $s_{2}$ & $b$ \\
\midrule
$Z$ & 0,000000 & 2,000000 & \cellcolor{blue!12}-9,000000 & 0,000000 & 12,000000 & 4,000000 & 100,000000 \\
$R_{1}$ & 0,000000 & 1,000000 & \cellcolor{blue!12}-6,000000 & 1,000000 & 6,000000 & -1,000000 & 20,000000 \\
$R_{2}$ & 1,000000 & 1,000000 & \cellcolor{blue!12}-1,000000 & 0,000000 & 1,000000 & 0,000000 & 5,000000 \\
\bottomrule
\end{tabular}
\end{table}

\section*{Resultados y Casos Especiales}
Estado del problema: \textbf{No acotado}.\\
\textbf{Problema no acotado:} la funci\'on objetivo puede crecer indefinidamente.\\
\vfill\smallskip\noindent Documento generado por Otro Simplex mas, de Esteban Secaida y Fabi\'an Bustos \end{document}
