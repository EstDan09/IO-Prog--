\documentclass[11pt]{article}
\usepackage[margin=2.2cm]{geometry}
\usepackage{booktabs}
\usepackage{array}
\usepackage{longtable}
\usepackage{float}
\usepackage[table]{xcolor}
\usepackage{hyperref}
\newcommand{\INF}{$\infty$}
\title{Proyecto 1 – Floyd–Warshall}
\date{\today}
\begin{document}
\maketitle
\section*{Descripción}
Reporte automático del algoritmo de Floyd–Warshall. Se muestran D(0) y P(0), todas las tablas intermedias D(k) y P(k) con cambios resaltados, y el resultado final.

\begin{table}[H]\centering
\caption{D(0) – matriz de distancias inicial}
\rowcolors{2}{white}{white}
\begin{tabular}{l r r r r}
\toprule
 & \textbf{A} & \textbf{B} & \textbf{C} & \textbf{D}\\\midrule
\textbf{A} & 0 & 8 & 4 & 5 \\
\textbf{B} & 5 & 0 & 5 & 6 \\
\textbf{C} & 5 & 2 & 0 & 4 \\
\textbf{D} & 9 & 3 & 7 & 0 \\
\bottomrule
\end{tabular}
\end{table}

\begin{table}[H]\centering
\caption{P(0) – matriz de siguiente salto inicial}
\rowcolors{2}{white}{white}
\begin{tabular}{l c c c c}
\toprule
 & \textbf{A} & \textbf{B} & \textbf{C} & \textbf{D}\\\midrule
\textbf{A} & - & B & C & D \\
\textbf{B} & A & - & C & D \\
\textbf{C} & A & B & - & D \\
\textbf{D} & A & B & C & - \\
\bottomrule
\end{tabular}
\end{table}

\begin{table}[H]\centering
\caption{D(1)}
\rowcolors{2}{white}{white}
\begin{tabular}{l r r r r}
\toprule
 & \textbf{A} & \textbf{B} & \textbf{C} & \textbf{D}\\\midrule
\textbf{A} & 0 & 8 & 4 & 5 \\
\textbf{B} & 5 & 0 & 5 & 6 \\
\textbf{C} & 5 & 2 & 0 & 4 \\
\textbf{D} & 9 & 3 & 7 & 0 \\
\bottomrule
\end{tabular}
\end{table}

\begin{table}[H]\centering
\caption{P(1)}
\rowcolors{2}{white}{white}
\begin{tabular}{l c c c c}
\toprule
 & \textbf{A} & \textbf{B} & \textbf{C} & \textbf{D}\\\midrule
\textbf{A} & \cellcolor{yellow!30}A & B & C & D \\
\textbf{B} & A & \cellcolor{yellow!30}A & C & D \\
\textbf{C} & A & B & \cellcolor{yellow!30}A & D \\
\textbf{D} & A & B & C & \cellcolor{yellow!30}A \\
\bottomrule
\end{tabular}
\end{table}

\begin{table}[H]\centering
\caption{D(2)}
\rowcolors{2}{white}{white}
\begin{tabular}{l r r r r}
\toprule
 & \textbf{A} & \textbf{B} & \textbf{C} & \textbf{D}\\\midrule
\textbf{A} & 0 & 8 & 4 & 5 \\
\textbf{B} & 5 & 0 & 5 & 6 \\
\textbf{C} & 5 & 2 & 0 & 4 \\
\textbf{D} & \cellcolor{yellow!30}8 & 3 & 7 & 0 \\
\bottomrule
\end{tabular}
\end{table}

\begin{table}[H]\centering
\caption{P(2)}
\rowcolors{2}{white}{white}
\begin{tabular}{l c c c c}
\toprule
 & \textbf{A} & \textbf{B} & \textbf{C} & \textbf{D}\\\midrule
\textbf{A} & A & B & C & D \\
\textbf{B} & A & A & C & D \\
\textbf{C} & A & B & A & D \\
\textbf{D} & \cellcolor{yellow!30}B & B & C & A \\
\bottomrule
\end{tabular}
\end{table}

\begin{table}[H]\centering
\caption{D(3)}
\rowcolors{2}{white}{white}
\begin{tabular}{l r r r r}
\toprule
 & \textbf{A} & \textbf{B} & \textbf{C} & \textbf{D}\\\midrule
\textbf{A} & 0 & \cellcolor{yellow!30}6 & 4 & 5 \\
\textbf{B} & 5 & 0 & 5 & 6 \\
\textbf{C} & 5 & 2 & 0 & 4 \\
\textbf{D} & 8 & 3 & 7 & 0 \\
\bottomrule
\end{tabular}
\end{table}

\begin{table}[H]\centering
\caption{P(3)}
\rowcolors{2}{white}{white}
\begin{tabular}{l c c c c}
\toprule
 & \textbf{A} & \textbf{B} & \textbf{C} & \textbf{D}\\\midrule
\textbf{A} & A & \cellcolor{yellow!30}C & C & D \\
\textbf{B} & A & A & C & D \\
\textbf{C} & A & B & A & D \\
\textbf{D} & B & B & C & A \\
\bottomrule
\end{tabular}
\end{table}

\begin{table}[H]\centering
\caption{D(4)}
\rowcolors{2}{white}{white}
\begin{tabular}{l r r r r}
\toprule
 & \textbf{A} & \textbf{B} & \textbf{C} & \textbf{D}\\\midrule
\textbf{A} & 0 & 6 & 4 & 5 \\
\textbf{B} & 5 & 0 & 5 & 6 \\
\textbf{C} & 5 & 2 & 0 & 4 \\
\textbf{D} & 8 & 3 & 7 & 0 \\
\bottomrule
\end{tabular}
\end{table}

\begin{table}[H]\centering
\caption{P(4)}
\rowcolors{2}{white}{white}
\begin{tabular}{l c c c c}
\toprule
 & \textbf{A} & \textbf{B} & \textbf{C} & \textbf{D}\\\midrule
\textbf{A} & A & C & C & D \\
\textbf{B} & A & A & C & D \\
\textbf{C} & A & B & A & D \\
\textbf{D} & B & B & C & A \\
\bottomrule
\end{tabular}
\end{table}

\section*{Distancias y rutas óptimas}
\begin{table}[H]\centering
\caption{D(final)}
\rowcolors{2}{white}{white}
\begin{tabular}{l r r r r}
\toprule
 & \textbf{A} & \textbf{B} & \textbf{C} & \textbf{D}\\\midrule
\textbf{A} & 0 & 6 & 4 & 5 \\
\textbf{B} & 5 & 0 & 5 & 6 \\
\textbf{C} & 5 & 2 & 0 & 4 \\
\textbf{D} & 8 & 3 & 7 & 0 \\
\bottomrule
\end{tabular}
\end{table}

\begin{table}[H]\centering
\caption{P(final)}
\rowcolors{2}{white}{white}
\begin{tabular}{l c c c c}
\toprule
 & \textbf{A} & \textbf{B} & \textbf{C} & \textbf{D}\\\midrule
\textbf{A} & A & C & C & D \\
\textbf{B} & A & A & C & D \\
\textbf{C} & A & B & A & D \\
\textbf{D} & B & B & C & A \\
\bottomrule
\end{tabular}
\end{table}

\subsection*{Listado de rutas (todas las parejas i \neq j)}
\begin{longtable}{llp{0.65\textwidth}}
\toprule
\textbf{Origen} & \textbf{Destino} & \textbf{Ruta óptima (con saltos)}\\\midrule\\[-1ex]
A & B & A → C → B (distancia = 6)\\
A & C & A → C (distancia = 4)\\
A & D & A → D (distancia = 5)\\
B & A & B → A (distancia = 5)\\
B & C & B → C (distancia = 5)\\
B & D & B → D (distancia = 6)\\
C & A & C → A (distancia = 5)\\
C & B & C → B (distancia = 2)\\
C & D & C → D (distancia = 4)\\
D & A & D → B → A (distancia = 8)\\
D & B & D → B (distancia = 3)\\
D & C & D → C (distancia = 7)\\
\bottomrule
\end{longtable}
\end{document}
