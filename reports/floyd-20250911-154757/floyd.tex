\documentclass[11pt]{article}
\usepackage[margin=2.2cm]{geometry}
\usepackage{booktabs}
\usepackage{array}
\usepackage{longtable}
\usepackage{float}
\usepackage[table]{xcolor}
\usepackage{hyperref}
\newcommand{\INF}{$\infty$}
\title{Proyecto 1 – Floyd–Warshall}
\date{\today}
\begin{document}
\maketitle
\section*{Descripción}
Reporte automático del algoritmo de Floyd–Warshall. Se muestran D(0) y P(0), todas las tablas intermedias D(k) y P(k) con cambios resaltados, y el resultado final.

\begin{table}[H]\centering
\caption{D(0) – matriz de distancias inicial}
\rowcolors{2}{white}{white}
\begin{tabular}{l r r r r}
\toprule
 & \textbf{A} & \textbf{B} & \textbf{C} & \textbf{D}\\\midrule
\textbf{A} & 0 & 8 & \INF & \INF \\
\textbf{B} & 4 & 0 & 5 & \INF \\
\textbf{C} & 3 & \INF & 0 & \INF \\
\textbf{D} & 1 & \INF & \INF & 0 \\
\bottomrule
\end{tabular}
\end{table}

\begin{table}[H]\centering
\caption{P(0) – matriz de siguiente salto inicial}
\rowcolors{2}{white}{white}
\begin{tabular}{l c c c c}
\toprule
 & \textbf{A} & \textbf{B} & \textbf{C} & \textbf{D}\\\midrule
\textbf{A} & - & B & - & - \\
\textbf{B} & A & - & C & - \\
\textbf{C} & A & - & - & - \\
\textbf{D} & A & - & - & - \\
\bottomrule
\end{tabular}
\end{table}

\begin{table}[H]\centering
\caption{D(1)}
\rowcolors{2}{white}{white}
\begin{tabular}{l r r r r}
\toprule
 & \textbf{A} & \textbf{B} & \textbf{C} & \textbf{D}\\\midrule
\textbf{A} & 0 & 8 & \INF & \INF \\
\textbf{B} & 4 & 0 & 5 & \INF \\
\textbf{C} & 3 & \cellcolor{yellow!30}11 & 0 & \INF \\
\textbf{D} & 1 & \cellcolor{yellow!30}9 & \INF & 0 \\
\bottomrule
\end{tabular}
\end{table}

\begin{table}[H]\centering
\caption{P(1)}
\rowcolors{2}{white}{white}
\begin{tabular}{l c c c c}
\toprule
 & \textbf{A} & \textbf{B} & \textbf{C} & \textbf{D}\\\midrule
\textbf{A} & \cellcolor{yellow!30}A & B & - & - \\
\textbf{B} & A & \cellcolor{yellow!30}A & C & - \\
\textbf{C} & A & - & \cellcolor{yellow!30}A & - \\
\textbf{D} & A & - & - & \cellcolor{yellow!30}A \\
\bottomrule
\end{tabular}
\end{table}

\begin{table}[H]\centering
\caption{D(2)}
\rowcolors{2}{white}{white}
\begin{tabular}{l r r r r}
\toprule
 & \textbf{A} & \textbf{B} & \textbf{C} & \textbf{D}\\\midrule
\textbf{A} & 0 & 8 & \cellcolor{yellow!30}13 & \INF \\
\textbf{B} & 4 & 0 & 5 & \INF \\
\textbf{C} & 3 & 11 & 0 & \INF \\
\textbf{D} & 1 & 9 & \cellcolor{yellow!30}14 & 0 \\
\bottomrule
\end{tabular}
\end{table}

\begin{table}[H]\centering
\caption{P(2)}
\rowcolors{2}{white}{white}
\begin{tabular}{l c c c c}
\toprule
 & \textbf{A} & \textbf{B} & \textbf{C} & \textbf{D}\\\midrule
\textbf{A} & A & B & \cellcolor{yellow!30}B & - \\
\textbf{B} & A & A & C & - \\
\textbf{C} & A & - & A & - \\
\textbf{D} & A & - & - & A \\
\bottomrule
\end{tabular}
\end{table}

\begin{table}[H]\centering
\caption{D(3)}
\rowcolors{2}{white}{white}
\begin{tabular}{l r r r r}
\toprule
 & \textbf{A} & \textbf{B} & \textbf{C} & \textbf{D}\\\midrule
\textbf{A} & 0 & 8 & 13 & \INF \\
\textbf{B} & 4 & 0 & 5 & \INF \\
\textbf{C} & 3 & 11 & 0 & \INF \\
\textbf{D} & 1 & 9 & 14 & 0 \\
\bottomrule
\end{tabular}
\end{table}

\begin{table}[H]\centering
\caption{P(3)}
\rowcolors{2}{white}{white}
\begin{tabular}{l c c c c}
\toprule
 & \textbf{A} & \textbf{B} & \textbf{C} & \textbf{D}\\\midrule
\textbf{A} & A & B & B & - \\
\textbf{B} & A & A & C & - \\
\textbf{C} & A & - & A & - \\
\textbf{D} & A & - & - & A \\
\bottomrule
\end{tabular}
\end{table}

\begin{table}[H]\centering
\caption{D(4)}
\rowcolors{2}{white}{white}
\begin{tabular}{l r r r r}
\toprule
 & \textbf{A} & \textbf{B} & \textbf{C} & \textbf{D}\\\midrule
\textbf{A} & 0 & 8 & 13 & \INF \\
\textbf{B} & 4 & 0 & 5 & \INF \\
\textbf{C} & 3 & 11 & 0 & \INF \\
\textbf{D} & 1 & 9 & 14 & 0 \\
\bottomrule
\end{tabular}
\end{table}

\begin{table}[H]\centering
\caption{P(4)}
\rowcolors{2}{white}{white}
\begin{tabular}{l c c c c}
\toprule
 & \textbf{A} & \textbf{B} & \textbf{C} & \textbf{D}\\\midrule
\textbf{A} & A & B & B & - \\
\textbf{B} & A & A & C & - \\
\textbf{C} & A & - & A & - \\
\textbf{D} & A & - & - & A \\
\bottomrule
\end{tabular}
\end{table}

\section*{Distancias y rutas óptimas}
\begin{table}[H]\centering
\caption{D(final)}
\rowcolors{2}{white}{white}
\begin{tabular}{l r r r r}
\toprule
 & \textbf{A} & \textbf{B} & \textbf{C} & \textbf{D}\\\midrule
\textbf{A} & 0 & 8 & 13 & \INF \\
\textbf{B} & 4 & 0 & 5 & \INF \\
\textbf{C} & 3 & 11 & 0 & \INF \\
\textbf{D} & 1 & 9 & 14 & 0 \\
\bottomrule
\end{tabular}
\end{table}

\begin{table}[H]\centering
\caption{P(final)}
\rowcolors{2}{white}{white}
\begin{tabular}{l c c c c}
\toprule
 & \textbf{A} & \textbf{B} & \textbf{C} & \textbf{D}\\\midrule
\textbf{A} & A & B & B & - \\
\textbf{B} & A & A & C & - \\
\textbf{C} & A & - & A & - \\
\textbf{D} & A & - & - & A \\
\bottomrule
\end{tabular}
\end{table}

\subsection*{Listado de rutas (todas las parejas i \neq j)}
\begin{longtable}{llp{0.65\textwidth}}
\toprule
\textbf{Origen} & \textbf{Destino} & \textbf{Ruta óptima (con saltos)}\\\midrule\\[-1ex]
A & B & A → B (distancia = 8)\\
A & C & A → B → C (distancia = 13)\\
A & D & No existe ruta.\\
B & A & B → A (distancia = 4)\\
B & C & B → C (distancia = 5)\\
B & D & No existe ruta.\\
C & A & C → A (distancia = 3)\\
C & B & No existe ruta.\\
C & D & No existe ruta.\\
D & A & D → A (distancia = 1)\\
D & B & No existe ruta.\\
D & C & No existe ruta.\\
\bottomrule
\end{longtable}
\end{document}
