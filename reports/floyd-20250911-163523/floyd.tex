\documentclass{article}
\usepackage[margin=2.2cm]{geometry}
\usepackage{booktabs}
\usepackage{array}
\usepackage{longtable}
\usepackage{float}
\usepackage[table]{xcolor}
\usepackage{hyperref}
\usepackage[utf8]{inputenc}
\usepackage{longtable,booktabs}
\usepackage{tikz}
\section*{Grafo de distancias}
\begin{tikzpicture}[every node/.style={circle, draw}, >=Stealth]
  \node (A) at (0*360/4:3) {A};
  \node (B) at (1*360/4:3) {B};
  \node (C) at (2*360/4:3) {C};
  \node (D) at (3*360/4:3) {D};
\end{tikzpicture}
\section*{Descripción}
Reporte automático del algoritmo de Floyd--Warshall. Se muestran D(0) y P(0), todas las tablas intermedias D(k) y P(k) con cambios resaltados, y el resultado final.

\begin{table}[H]\centering
\caption{D(0) -- matriz de distancias inicial}
\rowcolors{2}{white}{white}
\begin{tabular}{l r r r r}
\toprule
 & \textbf{A} & \textbf{B} & \textbf{C} & \textbf{D}\\\midrule
\textbf{A} & 0 & \INF & \INF & \INF \\
\textbf{B} & \INF & 0 & \INF & \INF \\
\textbf{C} & \INF & \INF & 0 & \INF \\
\textbf{D} & \INF & \INF & \INF & 0 \\
\bottomrule
\end{tabular}
\end{table}

\begin{table}[H]\centering
\caption{P(0) -- matriz de siguiente salto inicial}
\rowcolors{2}{white}{white}
\begin{tabular}{l c c c c}
\toprule
 & \textbf{A} & \textbf{B} & \textbf{C} & \textbf{D}\\\midrule
\textbf{A} & - & - & - & - \\
\textbf{B} & - & - & - & - \\
\textbf{C} & - & - & - & - \\
\textbf{D} & - & - & - & - \\
\bottomrule
\end{tabular}
\end{table}

\begin{table}[H]\centering
\caption{D(1)}
\rowcolors{2}{white}{white}
\begin{tabular}{l r r r r}
\toprule
 & \textbf{A} & \textbf{B} & \textbf{C} & \textbf{D}\\\midrule
\textbf{A} & 0 & \INF & \INF & \INF \\
\textbf{B} & \INF & 0 & \INF & \INF \\
\textbf{C} & \INF & \INF & 0 & \INF \\
\textbf{D} & \INF & \INF & \INF & 0 \\
\bottomrule
\end{tabular}
\end{table}

\begin{table}[H]\centering
\caption{P(1)}
\rowcolors{2}{white}{white}
\begin{tabular}{l c c c c}
\toprule
 & \textbf{A} & \textbf{B} & \textbf{C} & \textbf{D}\\\midrule
\textbf{A} & \cellcolor{yellow!30}A & - & - & - \\
\textbf{B} & - & \cellcolor{yellow!30}A & - & - \\
\textbf{C} & - & - & \cellcolor{yellow!30}A & - \\
\textbf{D} & - & - & - & \cellcolor{yellow!30}A \\
\bottomrule
\end{tabular}
\end{table}

\begin{table}[H]\centering
\caption{D(2)}
\rowcolors{2}{white}{white}
\begin{tabular}{l r r r r}
\toprule
 & \textbf{A} & \textbf{B} & \textbf{C} & \textbf{D}\\\midrule
\textbf{A} & 0 & \INF & \INF & \INF \\
\textbf{B} & \INF & 0 & \INF & \INF \\
\textbf{C} & \INF & \INF & 0 & \INF \\
\textbf{D} & \INF & \INF & \INF & 0 \\
\bottomrule
\end{tabular}
\end{table}

\begin{table}[H]\centering
\caption{P(2)}
\rowcolors{2}{white}{white}
\begin{tabular}{l c c c c}
\toprule
 & \textbf{A} & \textbf{B} & \textbf{C} & \textbf{D}\\\midrule
\textbf{A} & A & - & - & - \\
\textbf{B} & - & A & - & - \\
\textbf{C} & - & - & A & - \\
\textbf{D} & - & - & - & A \\
\bottomrule
\end{tabular}
\end{table}

\begin{table}[H]\centering
\caption{D(3)}
\rowcolors{2}{white}{white}
\begin{tabular}{l r r r r}
\toprule
 & \textbf{A} & \textbf{B} & \textbf{C} & \textbf{D}\\\midrule
\textbf{A} & 0 & \INF & \INF & \INF \\
\textbf{B} & \INF & 0 & \INF & \INF \\
\textbf{C} & \INF & \INF & 0 & \INF \\
\textbf{D} & \INF & \INF & \INF & 0 \\
\bottomrule
\end{tabular}
\end{table}

\begin{table}[H]\centering
\caption{P(3)}
\rowcolors{2}{white}{white}
\begin{tabular}{l c c c c}
\toprule
 & \textbf{A} & \textbf{B} & \textbf{C} & \textbf{D}\\\midrule
\textbf{A} & A & - & - & - \\
\textbf{B} & - & A & - & - \\
\textbf{C} & - & - & A & - \\
\textbf{D} & - & - & - & A \\
\bottomrule
\end{tabular}
\end{table}

\begin{table}[H]\centering
\caption{D(4)}
\rowcolors{2}{white}{white}
\begin{tabular}{l r r r r}
\toprule
 & \textbf{A} & \textbf{B} & \textbf{C} & \textbf{D}\\\midrule
\textbf{A} & 0 & \INF & \INF & \INF \\
\textbf{B} & \INF & 0 & \INF & \INF \\
\textbf{C} & \INF & \INF & 0 & \INF \\
\textbf{D} & \INF & \INF & \INF & 0 \\
\bottomrule
\end{tabular}
\end{table}

\begin{table}[H]\centering
\caption{P(4)}
\rowcolors{2}{white}{white}
\begin{tabular}{l c c c c}
\toprule
 & \textbf{A} & \textbf{B} & \textbf{C} & \textbf{D}\\\midrule
\textbf{A} & A & - & - & - \\
\textbf{B} & - & A & - & - \\
\textbf{C} & - & - & A & - \\
\textbf{D} & - & - & - & A \\
\bottomrule
\end{tabular}
\end{table}

\section*{Distancias y rutas óptimas}
\begin{table}[H]\centering
\caption{D(final)}
\rowcolors{2}{white}{white}
\begin{tabular}{l r r r r}
\toprule
 & \textbf{A} & \textbf{B} & \textbf{C} & \textbf{D}\\\midrule
\textbf{A} & 0 & \INF & \INF & \INF \\
\textbf{B} & \INF & 0 & \INF & \INF \\
\textbf{C} & \INF & \INF & 0 & \INF \\
\textbf{D} & \INF & \INF & \INF & 0 \\
\bottomrule
\end{tabular}
\end{table}

\begin{table}[H]\centering
\caption{P(final)}
\rowcolors{2}{white}{white}
\begin{tabular}{l c c c c}
\toprule
 & \textbf{A} & \textbf{B} & \textbf{C} & \textbf{D}\\\midrule
\textbf{A} & A & - & - & - \\
\textbf{B} & - & A & - & - \\
\textbf{C} & - & - & A & - \\
\textbf{D} & - & - & - & A \\
\bottomrule
\end{tabular}
\end{table}

\subsection*{Listado de rutas (todas las parejas i $\neq$ j)}
\begin{longtable}{llp{0.65\textwidth}}
\toprule
\textbf{Origen} & \textbf{Destino} & \textbf{Ruta óptima (con saltos)}\\\midrule
A & B & No existe ruta.\\
A & C & No existe ruta.\\
A & D & No existe ruta.\\
B & A & No existe ruta.\\
B & C & No existe ruta.\\
B & D & No existe ruta.\\
C & A & No existe ruta.\\
C & B & No existe ruta.\\
C & D & No existe ruta.\\
D & A & No existe ruta.\\
D & B & No existe ruta.\\
D & C & No existe ruta.\\
\bottomrule
\end{longtable}
\end{document}