\documentclass{article}
\usepackage[margin=2.2cm]{geometry}
\usepackage{booktabs}
\usepackage{array}
\usepackage{longtable}
\usepackage{float}
\usepackage[table]{xcolor}
\usepackage{hyperref}
\usepackage[utf8]{inputenc}
\usepackage{tikz}
\newcommand{\INF}{$\infty$}
\title{Proyecto 1 – Floyd–Warshall}
\author{Fabian Bustos - Esteban Secaida}
\date{\today}
\begin{document}
\maketitle
\section*{Grafo resultante}
\begin{center}
\begin{tikzpicture}[every node/.style={circle,draw,minimum size=8mm}]
\node (N0) at (0.00:4cm) {A};
\node (N1) at (90.00:4cm) {B};
\node (N2) at (180.00:4cm) {C};
\node (N3) at (270.00:4cm) {D};
\draw[->,>=stealth] (N0) -- (N1) node[midway,above] {2};
\draw[->,>=stealth] (N0) -- (N2) node[midway,above] {2};
\draw[->,>=stealth] (N0) -- (N3) node[midway,above] {3};
\draw[->,>=stealth] (N1) -- (N0) node[midway,above] {4};
\draw[->,>=stealth] (N1) -- (N2) node[midway,above] {5};
\draw[->,>=stealth] (N1) -- (N3) node[midway,above] {5};
\draw[->,>=stealth] (N2) -- (N0) node[midway,above] {5};
\draw[->,>=stealth] (N2) -- (N1) node[midway,above] {5};
\draw[->,>=stealth] (N2) -- (N3) node[midway,above] {5};
\draw[->,>=stealth] (N3) -- (N0) node[midway,above] {3};
\draw[->,>=stealth] (N3) -- (N1) node[midway,above] {1};
\draw[->,>=stealth] (N3) -- (N2) node[midway,above] {5};
\end{tikzpicture}
\end{center}
\section*{Descripción}
Reporte automático del algoritmo de Floyd--Warshall. Se muestran D(0) y P(0), todas las tablas intermedias D(k) y P(k) con cambios resaltados, y el resultado final.

\begin{table}[H]\centering
\caption{D(0) -- matriz de distancias inicial}
\rowcolors{2}{white}{white}
\begin{tabular}{l r r r r}
\toprule
 & \textbf{A} & \textbf{B} & \textbf{C} & \textbf{D}\\\midrule
\textbf{A} & 0 & 2 & 2 & 3 \\
\textbf{B} & 4 & 0 & 5 & 5 \\
\textbf{C} & 5 & 5 & 0 & 5 \\
\textbf{D} & 3 & 1 & 7 & 0 \\
\bottomrule
\end{tabular}
\end{table}

\begin{table}[H]\centering
\caption{P(0) -- matriz de siguiente salto inicial}
\rowcolors{2}{white}{white}
\begin{tabular}{l c c c c}
\toprule
 & \textbf{A} & \textbf{B} & \textbf{C} & \textbf{D}\\\midrule
\textbf{A} & - & B & C & D \\
\textbf{B} & A & - & C & D \\
\textbf{C} & A & B & - & D \\
\textbf{D} & A & B & C & - \\
\bottomrule
\end{tabular}
\end{table}

\begin{table}[H]\centering
\caption{D(1)}
\rowcolors{2}{white}{white}
\begin{tabular}{l r r r r}
\toprule
 & \textbf{A} & \textbf{B} & \textbf{C} & \textbf{D}\\\midrule
\textbf{A} & 0 & 2 & 2 & 3 \\
\textbf{B} & 4 & 0 & 5 & 5 \\
\textbf{C} & 5 & 5 & 0 & 5 \\
\textbf{D} & 3 & 1 & \cellcolor{yellow!30}5 & 0 \\
\bottomrule
\end{tabular}
\end{table}

\begin{table}[H]\centering
\caption{P(1)}
\rowcolors{2}{white}{white}
\begin{tabular}{l c c c c}
\toprule
 & \textbf{A} & \textbf{B} & \textbf{C} & \textbf{D}\\\midrule
\textbf{A} & \cellcolor{yellow!30}A & B & C & D \\
\textbf{B} & A & \cellcolor{yellow!30}A & C & D \\
\textbf{C} & A & B & \cellcolor{yellow!30}A & D \\
\textbf{D} & A & B & C & \cellcolor{yellow!30}A \\
\bottomrule
\end{tabular}
\end{table}

\begin{table}[H]\centering
\caption{D(2)}
\rowcolors{2}{white}{white}
\begin{tabular}{l r r r r}
\toprule
 & \textbf{A} & \textbf{B} & \textbf{C} & \textbf{D}\\\midrule
\textbf{A} & 0 & 2 & 2 & 3 \\
\textbf{B} & 4 & 0 & 5 & 5 \\
\textbf{C} & 5 & 5 & 0 & 5 \\
\textbf{D} & 3 & 1 & 5 & 0 \\
\bottomrule
\end{tabular}
\end{table}

\begin{table}[H]\centering
\caption{P(2)}
\rowcolors{2}{white}{white}
\begin{tabular}{l c c c c}
\toprule
 & \textbf{A} & \textbf{B} & \textbf{C} & \textbf{D}\\\midrule
\textbf{A} & A & B & C & D \\
\textbf{B} & A & A & C & D \\
\textbf{C} & A & B & A & D \\
\textbf{D} & A & B & C & A \\
\bottomrule
\end{tabular}
\end{table}

\begin{table}[H]\centering
\caption{D(3)}
\rowcolors{2}{white}{white}
\begin{tabular}{l r r r r}
\toprule
 & \textbf{A} & \textbf{B} & \textbf{C} & \textbf{D}\\\midrule
\textbf{A} & 0 & 2 & 2 & 3 \\
\textbf{B} & 4 & 0 & 5 & 5 \\
\textbf{C} & 5 & 5 & 0 & 5 \\
\textbf{D} & 3 & 1 & 5 & 0 \\
\bottomrule
\end{tabular}
\end{table}

\begin{table}[H]\centering
\caption{P(3)}
\rowcolors{2}{white}{white}
\begin{tabular}{l c c c c}
\toprule
 & \textbf{A} & \textbf{B} & \textbf{C} & \textbf{D}\\\midrule
\textbf{A} & A & B & C & D \\
\textbf{B} & A & A & C & D \\
\textbf{C} & A & B & A & D \\
\textbf{D} & A & B & C & A \\
\bottomrule
\end{tabular}
\end{table}

\begin{table}[H]\centering
\caption{D(4)}
\rowcolors{2}{white}{white}
\begin{tabular}{l r r r r}
\toprule
 & \textbf{A} & \textbf{B} & \textbf{C} & \textbf{D}\\\midrule
\textbf{A} & 0 & 2 & 2 & 3 \\
\textbf{B} & 4 & 0 & 5 & 5 \\
\textbf{C} & 5 & 5 & 0 & 5 \\
\textbf{D} & 3 & 1 & 5 & 0 \\
\bottomrule
\end{tabular}
\end{table}

\begin{table}[H]\centering
\caption{P(4)}
\rowcolors{2}{white}{white}
\begin{tabular}{l c c c c}
\toprule
 & \textbf{A} & \textbf{B} & \textbf{C} & \textbf{D}\\\midrule
\textbf{A} & A & B & C & D \\
\textbf{B} & A & A & C & D \\
\textbf{C} & A & B & A & D \\
\textbf{D} & A & B & C & A \\
\bottomrule
\end{tabular}
\end{table}

\section*{Distancias y rutas óptimas}
\begin{table}[H]\centering
\caption{D(final)}
\rowcolors{2}{white}{white}
\begin{tabular}{l r r r r}
\toprule
 & \textbf{A} & \textbf{B} & \textbf{C} & \textbf{D}\\\midrule
\textbf{A} & 0 & 2 & 2 & 3 \\
\textbf{B} & 4 & 0 & 5 & 5 \\
\textbf{C} & 5 & 5 & 0 & 5 \\
\textbf{D} & 3 & 1 & 5 & 0 \\
\bottomrule
\end{tabular}
\end{table}

\begin{table}[H]\centering
\caption{P(final)}
\rowcolors{2}{white}{white}
\begin{tabular}{l c c c c}
\toprule
 & \textbf{A} & \textbf{B} & \textbf{C} & \textbf{D}\\\midrule
\textbf{A} & A & B & C & D \\
\textbf{B} & A & A & C & D \\
\textbf{C} & A & B & A & D \\
\textbf{D} & A & B & C & A \\
\bottomrule
\end{tabular}
\end{table}

\subsection*{Listado de rutas (todas las parejas i $\neq$ j)}
\begin{longtable}{llp{0.65\textwidth}}
\toprule
\textbf{Origen} & \textbf{Destino} & \textbf{Ruta óptima (con saltos)}\\\midrule
A & B & A → B (distancia = 2)\\
A & C & A → C (distancia = 2)\\
A & D & A → D (distancia = 3)\\
B & A & B → A (distancia = 4)\\
B & C & B → C (distancia = 5)\\
B & D & B → D (distancia = 5)\\
C & A & C → A (distancia = 5)\\
C & B & C → B (distancia = 5)\\
C & D & C → D (distancia = 5)\\
D & A & D → A (distancia = 3)\\
D & B & D → B (distancia = 1)\\
D & C & D → C (distancia = 5)\\
\bottomrule
\end{longtable}
\end{document}