\documentclass[11pt]{article}
\usepackage[margin=1in]{geometry}
\usepackage{amsmath, amssymb}
\usepackage[table]{xcolor}
\usepackage{longtable}
\title{Proyecto 2: Problema de la Mochila}\date{\today}
\begin{document}
\begin{titlepage}
  \centering
  \vfill
  {\Huge Proyecto 2 : Problema de la Mochila}\par
  \vspace{1cm}
  {\Large Curso: Investigación de Operaciones}\par
  {\Large Semestre: II - 2025}\par
  \vfill
  {\Large Autores: Fabian Bustos - Esteban Secaida}\par
  \vspace{1cm}
  {\large Fecha: \today}\par
  \vfill
\end{titlepage}

\section*{Descripción}
Se resuelve el problema de la mochila en variante \textit{0/1} (capacidad $W=5$) con 4 objetos.\\
\subsection*{Problema ingresado}
Maximizar $Z = \sum_{i=1}^{4} v_i x_i$ \quad sujeto a $\sum_{i=1}^{4} w_i x_i \le 5$, $x_i \ge 0$ enteras, $x_i\in \{0,1\}$.
\\Datos:\\\
\begin{longtable}{r|lrrr}\# & Nombre & $w_i$ & $v_i$ & $q_i$\\\hline
1 & item & 1 & 1 & 1 \\
2 & item & 1 & 1 & 1 \\
3 & item & 1 & 1 & 1 \\
4 & item & 1 & 1 & 1 \\
\end{longtable}
\subsection*{Tabla de trabajo (DP)}
\setlength{\tabcolsep}{4pt}\renewcommand{\arraystretch}{1.1}
\begin{center}
\noindent\begin{tabular}{r|rrrrrr}\hline
$i\backslash W$ & 0 & 1 & 2 & 3 & 4 & 5 \\\hline
0 & \textcolor{black}{0} & \textcolor{black}{0} & \textcolor{black}{0} & \textcolor{black}{0} & \textcolor{black}{0} & \textcolor{black}{0} \\
1 & \textcolor{green!70!black}{0} & \textcolor{red!70!black}{1} & \textcolor{red!70!black}{1} & \textcolor{red!70!black}{1} & \textcolor{red!70!black}{1} & \textcolor{red!70!black}{1} \\
2 & \textcolor{green!70!black}{0} & \textcolor{green!70!black}{1} & \textcolor{red!70!black}{2} & \textcolor{red!70!black}{2} & \textcolor{red!70!black}{2} & \textcolor{red!70!black}{2} \\
3 & \textcolor{green!70!black}{0} & \textcolor{green!70!black}{1} & \textcolor{green!70!black}{2} & \textcolor{red!70!black}{3} & \textcolor{red!70!black}{3} & \textcolor{red!70!black}{3} \\
4 & \textcolor{green!70!black}{0} & \textcolor{green!70!black}{1} & \textcolor{green!70!black}{2} & \textcolor{green!70!black}{3} & \textcolor{red!70!black}{4} & \textcolor{red!70!black}{4} \\
\hline\end{tabular}
\end{center}
\subsection*{Solución óptima}
Valor óptimo $Z^* = 4$.\\
Solución 1: $x_{1}=1$ $x_{2}=1$ $x_{3}=1$ $x_{4}=1$ \\
\end{document}
