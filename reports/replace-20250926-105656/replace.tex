\documentclass[11pt]{article}
\usepackage[margin=2.5cm]{geometry}
\usepackage{booktabs}
\usepackage{hyperref}
\usepackage{amsmath}
\usepackage[T1]{fontenc}
\usepackage[utf8]{inputenc}
\usepackage[spanish]{babel}
\usepackage{tikz}
\usetikzlibrary{positioning}
\tikzset{every node/.style={circle, draw, minimum size=8mm}}
\begin{document}
\begin{titlepage}
\centering
{\Large Instituto Tecnol\'ogico de Costa Rica\\Escuela de Computaci\'on}\vspace{1cm}
{\huge Proyecto: Reemplazo de Equipos}\vspace{1cm}
{\large II Semestre 2025}\vspace{2cm}
{\large Estudiante(s):}\\Esteban Secaida (y equipo)\vfill
{\large Fecha: \today}
\end{titlepage}
\section*{Datos del Problema}
Costo inicial: $100.00$, Horizonte $T=10$, Vida \'util $L=3$.\\
\textbf{Sin ganancia por uso}.\\
\textbf{Sin inflaci\'on}.\\
\subsection*{Mantenimiento y Reventa por Edad}
\begin{tabular}{c|c|c}\toprule
Edad & Mant. & Reventa\\\midrule
1 & 100.00 & 400.00 \\
2 & 200.00 & 200.00 \\
3 & 300.00 & 600.00 \\
\bottomrule\end{tabular}
Se usa $C_{t,x}=\text{Compra}+\sum_{k=1}^{x-t}(\text{Mant}(k))\cdot(1+i)^{k-1}-\text{Reventa}(x-t)\cdot(1+i)^{x-t-1}$ si hay inflaci\'on.\\
\section*{Tabla de $C_{t,x}$}
Entradas v\'alidas con $t<x\le\min(t+L,T)$.

\begin{tabular}{c|c|c}\toprule
 t & x & $C_{t,x}$ \\\midrule
0 & 1 & -200.00 \\
0 & 2 & 200.00 \\
0 & 3 & 100.00 \\
1 & 2 & -200.00 \\
1 & 3 & 200.00 \\
1 & 4 & 100.00 \\
2 & 3 & -200.00 \\
2 & 4 & 200.00 \\
2 & 5 & 100.00 \\
3 & 4 & -200.00 \\
3 & 5 & 200.00 \\
3 & 6 & 100.00 \\
4 & 5 & -200.00 \\
4 & 6 & 200.00 \\
4 & 7 & 100.00 \\
5 & 6 & -200.00 \\
5 & 7 & 200.00 \\
5 & 8 & 100.00 \\
6 & 7 & -200.00 \\
6 & 8 & 200.00 \\
6 & 9 & 100.00 \\
7 & 8 & -200.00 \\
7 & 9 & 200.00 \\
7 & 10 & 100.00 \\
8 & 9 & -200.00 \\
8 & 10 & 200.00 \\
9 & 10 & -200.00 \\
\bottomrule\end{tabular}
\section*{Programaci\'on Din\'amica: $G(t)$ y Siguientes}
\begin{tabular}{c|c|l}\toprule
 t & $G(t)$ & Siguientes \\\midrule
0 & -2000.00 & 1 \\
1 & -1800.00 & 2 \\
2 & -1600.00 & 3 \\
3 & -1400.00 & 4 \\
4 & -1200.00 & 5 \\
5 & -1000.00 & 6 \\
6 & -800.00 & 7 \\
7 & -600.00 & 8 \\
8 & -400.00 & 9 \\
9 & -200.00 & 10 \\
10 & 0.00 &  \\
\bottomrule\end{tabular}
\section*{Todos los planes \`optimos}
Costo m\'inimo total: $G(0)=\mathbf{-2000.00}$.\\
\noindent Ruta 1: \texttt{0 -> 1 -> 2 -> 3 -> 4 -> 5 -> 6 -> 7 -> 8 -> 9 -> 10}\\
\subsection*{Ruta óptima 1}
\begin{tikzpicture}[->, >=stealth, node distance=2cm]
\node (R0Y1) at (0,0) {1};
\node (R0Y2) at (2,0) {2};
\node (R0Y3) at (4,0) {3};
\node (R0Y4) at (6,0) {4};
\node (R0Y5) at (8,0) {5};
\node (R0Y6) at (10,0) {6};
\node (R0Y7) at (12,0) {7};
\node (R0Y8) at (14,0) {8};
\node (R0Y9) at (16,0) {9};
\node (R0Y10) at (18,0) {10};
\draw[->] (R0Y1) -- (R0Y2);
\draw[->] (R0Y2) -- (R0Y3);
\draw[->] (R0Y3) -- (R0Y4);
\draw[->] (R0Y4) -- (R0Y5);
\draw[->] (R0Y5) -- (R0Y6);
\draw[->] (R0Y6) -- (R0Y7);
\draw[->] (R0Y7) -- (R0Y8);
\draw[->] (R0Y8) -- (R0Y9);
\draw[->] (R0Y9) -- (R0Y10);
\end{tikzpicture}
\bigskip
\end{document}
